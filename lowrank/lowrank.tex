
\chapter[Behavior of the SVD]{Behavior of the SVD in low-rank plus noise models}


\newcommand{\ba}{\boldsymbol{a}}      
\newcommand{\be}{\boldsymbol{e}}      
\newcommand{\bo}{\boldsymbol{o}}      
\newcommand{\bu}{\boldsymbol{u}}      
\newcommand{\bv}{\boldsymbol{v}}
\newcommand{\by}{\boldsymbol{y}}
\newcommand{\bz}{\boldsymbol{z}}

\newcommand{\tPsi}{\tilde \Psi}       % tilde symbols

\newcommand{\Q}{\mathcal{Q}}
\newcommand{\U}{\mathcal{U}}          % script symbols
\newcommand{\V}{\mathcal{V}}
\newcommand{\W}{\mathcal{W}}


\newcommand{\Thalf}{\frac{T}{2}}
\newcommand{\fourth}{\frac{1}{4}}


\section{Introduction}

Many modern data sets involve simultaneous measurements of a large number of
variables. Some financial applications including portfolio selection involve
looking at the market prices of hundreds or thousands of stocks and their
evolution over time \cite{markowitz1952ps}. Microarray studies
\cite{biotechnol1995sms} involve measuring the expression levels of thousands of
genes simultaneously. Text processing \cite{manning1999fsn} involves counting
the appearances of thousands of words on thousands of documents. In all of these
applications, it is natural to suppose that even though the data are
high-dimensional, their dynamics are driven by a relatively small number of
latent factors.

Under the hypothesis that one ore more latent factors explain the behavior
of a data set, principal component analysis (PCA) \cite{jolliffe2002pca} is a
popular method for estimating these latent factors.  When the dimensionality
of the data is small relative to the sample size, Anderson's 1963 paper
\cite{anderson1963atp} gives a complete treatment of how the procedure
behaves.  Unfortunately, his results do not apply when the sample size is
comparable to the dimensionality.

A further complication with many modern data sets is that it is no longer
appropriate to assume the observations are \iid Also, sometimes it is
difficult to distinguish between ``observation'' and ``variable''. We call such
data ``transposable''. A microarray study involving measurements of $p$ genes
for $n$ different patients can be considered transposable: we can either treat
each gene as a measurement of the patient, or we can treat each patient as a
measurement of the gene. There are correlations both between the genes
\emph{and} between the patients, so in fact both interpretations are relevant
\cite{efron2008smi}.

One can study latent factor models in a transpose-agnostic way be considering
generative models of the form $\mX = \mU \mD \mV^\trans + \mE$. Here, $\mX$ is
the $n \times p$ observed data matrix. The unobserved row and column factors
are given by $\mU$ and $\mV$, respectively, matrices with $k$ orthonormal
columns each, with $k \ll \min(n,p)$. The strengths of the factors are given
in the $k\times k$ diagonal matrix $\mD$, and $\mE$ is a matrix of noise. A
natural estimator for the latent factor term $\mU \mD \mV^\trans$ can be
gotten from truncating the singular value decomposition (SVD)
\cite{golub1996mc} of $\mX$. The goal of this paper is to study the behavior
of the SVD when $n$ and $p$ both tend to infinity, with their ratio tending to
a nonzero constant.

In an upcoming paper, Onatski~\cite{onatski} gives a very thorough treatment
of these models. He assumes that the elements of $E$ are \iid Gaussians, that
$\sqrt{n} ( \mU^\trans \mU - \mI_k) $ tends to a multivariate Gaussian
distribution, and that $\mV$ and $\mD$ are both nonrandom.

Slight variations of the above model have been analyzed by
Baik~\&~Silverstein~\cite{baik2006els}, and by Paul~\cite{paul2007ase}. These
authors look at matrices of the form $\mX = \mT^{1/2} \mZ$, where $\mZ$ is a
matrix with iid entries, and $\mT$ is a ``spiked'' covariance matrix, i.e. a
positive-definite matrix with a few large eigenvalues. Baik \& Silverstein
prove quite general results using the Stieltjes transform only assuming finite
fourth moments, but they limit their study to the almost sure limits of the
top singular values of $\mX$. Paul gives the limiting distributions of the
sample singular values as well as the singular vectors, but he assumes that
the elements of $\mZ$ are independent standard Gaussians. Later, Chen and
coauthors \cite{chen2009ppc} generalized his results to allow for correlated
Gaussian loadings of the factors.

The contributions of this chapter are twofold. First, we work under a
transpose-agnostic generative model that allows randomness in all three of
$\mU$, $\mD$, and $\mV$. Second, we give a more complete picture of the
almost-sure limits of the sample singular vectors, taking into account their
projections onto the space orthogonal to the population singular vectors.

We describe the results in Section 2. The rest of the paper is dedicated to
proving the two main theorems. We owe a substantial debt to Onatski's work.
Although some significant details below are different, the general outline and
the main points of the argument are the same.

\section{Assumptions, notation, and main results}

Here we make explicit what the model and assumptions are, and we present our
main results.

\subsection{Assumptions and notation}

We will work sequences of matrices indexed by $n$, with
\[
    \mX_n = \sqrt{n} \mU_n \mD_n \mV_n^\trans + \mE_n.
\]
We denote by $\sqrt{n} \mU_n \mD_n \mV_n^\trans$ the ``signal'' part of the
matrix $\mX_n$ and $\mE_n$ the ``noise'' part. We will often refer to $\mU_n$
and $\mV_n$ as the left and right factors of $\mX_n$, and the matrix $\mD_n$
will be called the matrix of normalized factor strengths. The first two
assumptions concern the signal part:

\begin{assumption}\label{A:factors}
    The factors $\mU_n$ and $\mV_n$ are random matrices of dimensions
    $n \times k$ and $p \times k$, respectively.  The number of factors, $k$,
    is a fixed constant.  The factors are normalized so that
    $\mU_n^\trans \mU_n =  \mV_n^\trans \mV_n = \mI_k$, where $\mI_k$ denotes 
    the $k \times k$ identity matrix.  The ratio $p/n = \gamma$ is a fixed 
    nonzero scalar in the range $(0,\infty)$.
\end{assumption}
\noindent

\begin{assumption}\label{A:sizes}
    The matrix of factor strengths, $\mD_n$, is of size $k\times k$ and 
    diagonal, with 
    \(
        \mD_n = \diag \left( d_{n,1}, d_{n,2}, \ldots, d_{n,k} \right)
    \)
    and
    $d_{n,1} > d_{n,2} > \cdots > d_{n,k} > 0$.  The matrix $\mD_n$ converges
    almost surely to a deterministic matrix 
    $\mD = \diag( \mu_1^{1/2}, \mu_2^{1/2}, \ldots, \mu_k^{1/2})$ with
    $\mu_1 > \mu_2 > \cdots > \mu_k > 0$. Moreover, the vector
    \(
        \sqrt{n} ( d_{n,1}^2 - \mu_1, d_{n,1}^2 - \mu_2, 
                   \ldots, 
                   d_{n,K}^2 - \mu_k )
    \)
    converges in distribution to a mean-zero multivariate normal with 
    covariance matrix $\mSigma$ having entries $\Sigma_{ij} = \sigma_{ij}$
    (possibly degenerate).
\end{assumption}
\noindent
The next assumption concerns the noise part:

\begin{assumption}\label{A:noise}
    The noise matrix $\mE_n$ is an $n\times p$ matrix with entries 
    $E_{n,ij}$ independent $\Normal(0,\, \sigma^2)$ random variables, also
    independent of $\mU_n$, $\mD_n$, and $\mV_n$.
\end{assumption}
\noindent

We need to introduce some more notation. We denote the columns of $\mU_n$ and
$\mV_n$ by $\vU_{n,1}, \vU_{n,2}, \ldots, \vU_{n,k}$ and $\vV_{n,1},
\vV_{n,2}, \ldots, \vV_{n,k}$, respectively. We let 
\[
    \mX_n \approx \sqrt{n} \mhU_n \mhD_n \mhV_n^\trans
\]
be the singular value decomposition of
$\mX_n$ truncated to $k$ terms, where $\mhD_n = \diag( \hat \mu_{n,1}^{1/2},
\hat \mu_{n,2}^{1/2}, \ldots, \hat \mu_{n,k}^{1/2})$ and the columns of $\mhU$ and $\mhV$ are given
by $\vhU_{n,1}, \vhU_{n,2} \ldots, \vhU_{n,k}$ and $\vhV_{n,1}, \vhV_{n,2},
\ldots, \vhV_{n,k}$, respectively.


\subsection{Main results}

We are now in a position to say what the results are.  There are two main
theorems, one concerning the sample singular values and the other concerning
the sample singular vectors.  First we give the result about the singular
values.
\begin{theorem}\label{T:values}
    Under Assumptions~\ref{A:factors}~--~\ref{A:noise},
    the vector $\hat \vmu = (\hat \mu_1, \hat \mu_2, \ldots, \hat \mu_k)$
    converges almost surely to 
    $\tilde \vmu = (\tilde \mu_1, \tilde \mu_2, \ldots, \tilde \mu_k)$,
    where
    \[
        \tilde \mu_k
        =
        \begin{cases}
            \left( \mu_i + \sigma^2 \right)
            \left( 1 + \gamma \sigma^2 / \mu_i \right)
                &\text{when $\mu_i > \sqrt{\gamma} \sigma^2$}, \\
            \sigma^2 \left( 1 + \sqrt{\gamma} \right)^2
                &\text{otherwise.}
        \end{cases}
    \]
    Moveover, 
    \(
        \sqrt{n} (\hat \vmu - \tilde \vmu)
    \)
    converges in distribution to a (possibly degenerate) multivariate normal
    with covariance matrix $\mtSigma$ whose $ij$ element is given by
    \[
        \tilde \sigma_{ij}
        \equiv
        \begin{cases}
            \begin{aligned}
                \sigma_{ij}
                &\left(
                    1 - \gamma \sigma^4 / \mu_i^2
                \right)
                \left(
                    1 - \gamma \sigma^4 / \mu_j^2
                \right)
                \\
                &+ \delta_{ij}
                2
                \sigma^2
                \left(
                    2 \mu_i + \sigma^2 + \gamma \sigma^2
                \right)
                \left(
                    1 - \gamma \sigma^4 / \mu_i^2
                \right)
            \end{aligned}
                &\text{when $\mu_i, \, \mu_j > \gamma^{1/2} \sigma^2$,} \\
            0
                &\text{otherwise.}
        \end{cases}
    \]
    When $\sigma_{ii} = 2 \mu_i^2$, 
    and $\mu_i > \sqrt{\gamma} \sigma^2$, the variance of the $i$th
    component simplifies to
    \(
        \tilde \sigma_{ii} 
        = 
        2 (\mu_i + \sigma^2)^2 \left( 1 - \gamma \sigma^4 / \mu_i^2 \right).
    \)
\end{theorem}

Next, we give the result for the singular vectors:
\begin{theorem}\label{T:vectors}
    Say Assumptions~\ref{A:factors}~--~\ref{A:noise} hold.  Then the
    $k\times k$ matrix $\mR_n \equiv \mV_n^\trans \mhV_n$ converges almost 
    surely to a matrix $\mR = \diag(r_1, r_2, \ldots, r_k)$, where
    \begin{equation}
        |r_i|^2
        =
        \begin{cases}
            \frac{ 1 - \gamma \sigma^4 / \mu_i^2 }
                 { 1 + \gamma \sigma^2 / \mu_i  }
            &\text{when $\mu_i > \sqrt{\gamma} \sigma^2$,} \\
            0
            &\text{otherwise.}
        \end{cases}
    \end{equation}
\end{theorem}


\subsection{Notes}

Some discussion of the assumptions and the results are in order:

\begin{enumerate}
    
\item
Assumptions~\ref{A:factors}~and~\ref{A:sizes} are simpler than the assumptions given in many other papers while still being quite general.  For example, Paul's ``spiked'' covariance model has data of the form
\[
    \mX = \mZ \mTheta^\trans + \mE,
\]
where $\mTheta$ is an $p \times k$ matrix of factors and $\mZ$ is an $n\times k$ matrix of factor loadings whose rows are \iid multivariate $\Normal(0,\, \mC)$ random variables for covariance matrx $\mC$ having eigen-decomposition $\mC = \mQ \mLambda \mQ^\trans$.  Letting $\mZ = \sqrt{n} \mhP \mhLambda \mhQ^\trans$ be the SVD of $\mZ$, Anderson's results \cite{anderson1963atp} give us that $\mhLambda$ and $\mhQ$ converge almost surely to $\mLambda$ and $\mQ$, respectively, and that the diagonal elements of $\sqrt{n} (\mhLambda - \mLambda)$ converge to a mean-zero multivariate normal whenever $\mC$ has no repeated eigenvalues.  If we define $\mU = \mhP$, $\mD = \mhLambda^\half$, and $\mV = \mTheta \mhQ$, then $\mX = \sqrt{n} \mU \mD \mV^\trans + \mE$, where the factors satisfy Assumptions~\ref{A:factors}~and~\ref{A:sizes}.  Dropping the normality assumption on the rows of $\mZ$ poses no problem.  Moreover, we can suppose instead of \iid that the rows of $\mZ$ be a martingale difference array with well-behaved low-order moments and still perform a transformation of the variables to get factors of the form we need for Theorems~\ref{T:values}~and~\ref{T:vectors}.

\item
There is a sign-indeterminancy in the sample and population singular vectors. Since we only give results for the squares $\mR_n^\trans \mR$ the signs do not matter and we choose them arbitrarily.

\item 
Most of the results in Theorems~\ref{T:values}~and~\ref{T:vectors} can be gotten from Onatski's results \cite{onatski}.   However, Onatski does not show that $\sqrt{n} (\hat \mu_i - \tilde \mu_i) \toP 0$ when $\mu_i$ is below the critical threshold.  Furthermore, Onatski proves convergence in probability, not almost sure convergence.

\end{enumerate}

\section{Preliminaries}\label{S:lowrank-preliminaries}

Without loss of generality we will assume that $\sigma^2 = 1$.  For now, we will also assume that $\gamma \geq 1$.

\subsection{Change of basis}

A convenient choice of basis will make it easier to study the SVD of $\mX_n$.  Define $\mU_{n,1} = \mU_n$, and choose $\mU_{n,2}$ so that
\(
    \left( 
    \begin{matrix}
        \mU_{n,1} & \mU_{n,2}
    \end{matrix}
    \right)
\) 
is an orthogonal matrix. Similarly, put $\mV_{n,1} = \mV_n$ and choose
$\mV_{n,2}$ so that
\(
    \left(
    \begin{matrix}
        \mV_{n,1} & \mV_{n,2}
    \end{matrix}
    \right)
\)
is orthogonal.  If we define
\(
    \mtE_{n,ij} = \mU_{n,i}^\trans \mE_n \mV_{n,j}
\)
and
\(
    \mX_{n,ij} = \mU_{n,i}^\trans \mX_n \mV_{n,j},
\)
then in block form,
\[
    \left( 
    \begin{matrix}
        \mU_{n,1}^\trans \\
        \mU_{n,2}^\trans
    \end{matrix}
    \right)
    \mX_n
    \left(
    \begin{matrix}
        \mV_{n,1} & \mV_{n,2}
    \end{matrix}
    \right)
    =
    \left(
    \begin{matrix}
        \sqrt{n} \mD_n + \mtE_{n,11} & \mtE_{n,12} \\
        \mtE_{n,21}                  & \mtE_{n,22}
    \end{matrix}
    \right).
\]
Because Gaussian white noise is orthogonally invariant, the
matrices $\mtE_{n,ij}$ are all independent with \iid $\Normal(0, \, 1)$ elements.  Let
\[
    \mtE_{n,22}
    =
    \sqrt{n}
    \left(
    \begin{matrix}
        \mO_{n,1} & \mO_{n,2}
    \end{matrix}
    \right)
    \left(
    \begin{matrix}
        \mLambda_n^{1/2} \\
        0
    \end{matrix}
    \right)
    \mP_{n}^\trans
\]
be the SVD of $\mtE_{n,22}$, with
\(
    \mLambda_n
    =
    \diag \left(
        \lambda_{n,1}, 
        \lambda_{n,2}, 
        \ldots
        \lambda_{n,p-k}
    \right)
\)
.  Note that 
\(
    \mtE_{n,22}^\trans \mtE_{n,22}
    \sim
    \Wishart_{p-k} \left( n-k, \mI_{p-k} \right).
\)
Define
\begin{align*}
    \mtX_n
        &=
            \left(
            \begin{matrix}
                \mI_k & 0 \\
                0  & \mO_{n,1}^\trans \\
                0  & \mO_{n,2}^\trans
            \end{matrix}
            \right)
            \left(
            \begin{matrix}
                \sqrt{n} \mD_n + \mtE_{n,11} & \mtE_{n,12} \\
                \mtE_{n,21}                  & \mtE_{n,22}
            \end{matrix}
            \right)
            \left(
            \begin{matrix}
                \mI_k & 0 \\
                0     & \mP_n
            \end{matrix}
            \right) \\
        &=
            \left(
            \begin{matrix}
                \sqrt{n} \mD_n + \mE_{n,11} & \mE_{n,12} \\
                \mE_{n,21}                  & \sqrt{n} \mLambda^{1/2}_n \\
                \mE_{n,31}                  & 0
            \end{matrix}
            \right),
\end{align*}
where 
$\mE_{n,11} = \mtE_{n,11}$, 
$\mE_{n,12} = \mtE_{n,12} \mP_n$,
$\mE_{n,21} = \mO_{n,1}^\trans \mtE_{n,21}$, and
$\mE_{n,31} = \mO_{n,2}^\trans \mtE_{n,31}$.
Finally, let
\[
    \mtX_n
    \approx
    \mtU_n \mtD_n \mtV_n
\]
be the SVD of $\mtX_n$, truncated to $k$ terms.  Lastly, put the left and right singular vectors in block form as
\[
    \mtU_n
    =
    \left(
    \begin{matrix}
        \mtU_{n,1} \\
        \mtU_{n,2}
    \end{matrix}
    \right)
\]
and
\[
    \mtV_n
    =
    \left(
    \begin{matrix}
        \mtV_{n,1} \\
        \mtV_{n,2}
    \end{matrix}
    \right),
\]
where $\mtU_{n,1}$ and $\mtV_{n,1}$ both $k\times k$ matrices.

We have gotten to $\mtX_n$ via an orthogonal change of basis applied to $\mX_n$.  By carefully choosing this basis, we have assured that:
\begin{enumerate}
    \item The blocks of $\mtX_n$ are all independent.
    \item The elements of the matrices $\mE_{n,ij}$ are 
        \iid $\Normal\left( 0, 1 \right)$.
    \item The elements 
        $\{ n \lambda_{n,1}, n \lambda_{n,2}, \ldots, 
            n \lambda_{n,p-k} \}$ are eigenvalues from
        a white Wishart matrix with $n-k$ degrees of freedom.
    \item $\mtX_n$ and $\mX_n$ have the same singular values.  This implies
        that $\mhD_n = \mtD_n$.
    \item The left singular vectors of $\mX_n$ can be recovered from the
        left singular vectors of $\mtX_n$ via multiplication by an orthogonal
        matrix.  The same holds for the right singular vectors.
    \item The dot product matrix $\mU_n^\trans \mhU_n$ is equal to
        $\mtU_{n,1}$.
        Similarly, $\mV_n^\trans \mhV_n = \mtV_{n,1}$.
\end{enumerate}

\subsection{Perturbation expansion}

We set $\mS_n = \frac{1}{n} = \mtX_n^\trans \mtX_n$.  The eigenvalues and eigenvectors of $\mS_n$ are the squares of the singular values of $\mtX_n$ and its right singular vectors, respectively.  In block form, we have
\[
    \mS_n
    =
    \left(
    \begin{matrix}
        \mS_{n,11} & \mS_{n,12} \\
        \mS_{n,21} & \mS_{n,22}
    \end{matrix}
    \right),
\]
where
\begin{align*}
    \begin{split}
    \mS_{n,11}
        &=
            \mD_n^2 
            + 
            \frac{1}{\sqrt{n}} 
            \left( 
                \mD_n \mE_{n,11} + \mE_{n,11}^\trans \mD_n
            \right) \\
            &\phantom{= \mD_n^2} +
            \frac{1}{n}
            \left(
                \mE_{n,11}^\trans \mE_{n,11}
                +
                \mE_{n,21}^\trans \mE_{n,21}
                +
                \mE_{n,31}^\trans \mE_{n,31}
            \right),
    \end{split} \\
    \mS_{n,12}
        &=
            \frac{1}{\sqrt{n}}
            \left(
                \mD_n \mE_{n,12} + \mE_{n,21}^\trans \mLambda_n^{1/2}
            \right)
            +
            \frac{1}{n}
            \mE_{n,11}^\trans \mE_{n,12}, \\
    \mS_{n,21}
        &=
            \frac{1}{\sqrt{n}}
            \left(
                \mE_{n,12}^\trans \mD_n
                +
                \mLambda_n^{1/2} \mE_{n,21}
            \right)
            +
            \frac{1}{n}
            \mE_{n,12}^\trans \mE_{n,11}, \\
    \mS_{n,22}
        &=
            \mLambda_n
            +
            \frac{1}{n}
            \mE_{n,12}^\trans \mE_{n,12}.
\end{align*}
We write $\mS_n$ as a sum of two terms.  First, let
\[
    \mS_n^0
    =
    \left(
    \begin{matrix}
        \mS_{n,11}^0 & \mS_{n,12}^0 \\
        \mS_{n,21}^0 & \mS_{n,22}^0
    \end{matrix}
    \right),
\]
where
\begin{align*}
    \mS_{n,11}^0
        &=
            \mD_n^2 + \mI_k, \\
    \mS_{n,12}^0
        &=
            \frac{1}{\sqrt{n}} 
            \left( 
                \mD_n \mE_{n,12}
                +
                \mE_{n,21}^\trans \mLambda_n^{1/2} 
            \right), \\
    \mS_{n,21}^0
        &=
            \frac{1}{\sqrt{n}} 
            \left( 
                \mE_{n,12}^\trans \mD_n
                +
                \mLambda_n^{1/2} \mE_{n,21}
                \right), \\
    \mS_{n,22}^0
        &=
            \mLambda_n 
            +
            \frac{1}{n}
            \mE_{n,12}^\trans \mE_{n,12}.
\end{align*}
Next, put
\[
    \mS_n^1
    =
    \left(
    \begin{matrix}
        \mS_{n,11}^1 & \mS_{n,12}^1 \\
        \mS_{n,21}^1 & 0
    \end{matrix}
    \right),
\]
with
\begin{align*} 
    \begin{split}   
    \mS_{n,11}^1
        &=
            \mD_n \mE_{n,11}
            +
            \mE_{n,11}^\trans \mD_n \\
            &\qquad +
            \frac{1}{\sqrt{n}} 
            \left(
                \mE_{n,11}^\trans \mE_{n,11}
                +
                \mE_{n,21}^\trans \mE_{n,21}
                +
                \mE_{n,31}^\trans \mE_{n,31}
            \right)
            -
            \sqrt{n} \mI_k,
    \end{split} \\
    \mS_{n,12}^1
        &= \frac{1}{\sqrt{n}} \mE_{n,11}^\trans \mE_{n,12}, \\
    \mS_{n,21}^1
        &= \frac{1}{\sqrt{n}} \mE_{n,12}^\trans \mE_{n,11}.
\end{align*}

We have that
\[
    \mS_n = \mS_n^0 + \frac{1}{\sqrt{n}} \mS_n^1.
\]
Since $\| \mS_n^1 \|_\Frob = \OhP( k )$,
it turns out that the eigenvalues and eigenvectors of $\mS_n$ are small
perturbations of the corresponding properties of $\mS_n^0$.  We will explore this further in section TODO.

\subsection{The secular equation}

We first study the eigendecomposition of $\mS_n^0$.  If 
\(
    \vx
    = 
    \left( 
    \begin{matrix}
        \vx_{1} \\
        \vx_{2}
    \end{matrix}
    \right)
\)
is an eigenvector of $\mS_n^0$ with eigenvalue $\mu$, then
\[
    \left(
    \begin{matrix}
        \mS_{n,11}^0 & \mS_{n,12}^0 \\
        \mS_{n,21}^0 & \mS_{n,22}^0
    \end{matrix}
    \right)
    \left( 
    \begin{matrix}
        \vx_{1} \\
        \vx_{2}
    \end{matrix}
    \right)
    =
    \mu
    \left( 
    \begin{matrix}
        \vx_{1} \\
        \vx_{2}
    \end{matrix}
    \right).
\]
If $\mu$ is not an eigenvalue of $\mS_{n,22}^0$, then we get
\begin{subequations}
\begin{equation}\label{E:x2-from-x1}
    \vx_2
    =
    -
    \left(
        \mS_{n,22}^0
        -
        \mu
        \mI_{p-k}
    \right)^{-1}
    \mS_{n,21}^0 \,
    \vx_1
\end{equation}
and
\begin{equation}\label{E:secular-nof}
    \left(
        \mS_{n,11}^0
        -
        \mu
        \mI_k
        -
        \mS_{n,12}^0
        \left(
            \mS_{n,22}^0
            -
            \mu
            \mI_{p-k}
        \right)^{-1}
        \mS_{n,21}^0
    \right)
    \vx_1
    =
    0.
\end{equation}
\end{subequations}
Conversely, if $(\mu, \vx_1)$ is a pair that solves~\eqref{E:secular-nof}, then
\begin{equation}\label{E:x-from-x1}
    \vx
    =
    \left(
    \begin{matrix}
        \vx_1 \\
        -
        \left(
            \mS_{n,22}^0
            -
            \mu
            \mI_{p-k}
        \right)^{-1}
        \mS_{n,21}^0 \,
        \vx_1
    \end{matrix}
    \right)
\end{equation}
is an eigenvector of $\mS_n^0$ with eigenvalue $\mu$.  We define
\begin{equation}\label{E:secular-f}
    f_n ( \mu, \vx_1 )
    =
    \left(
        \mS_{n,11}^0
        -
        \mu
        \mI_k
        -
        \mS_{n,12}^0
        \left(
            \mS_{n,22}^0
            -
            \mu
            \mI_{p-k}
        \right)^{-1}
        \mS_{n,21}^0
    \right)
    \vx_1
\end{equation}
and refer to the equation $f_n(\mu, \vx_1) = 0$ as the \emph{secular equation}.

\subsection{Outline of the proof}

We first show that with probability one, $\mS_n^0$ and $\mS_{n,22}^0$ have no eigenvalues in common.  This implies that all of the eigenvalues of $\mS_n^0$ are solutions to the secular equation.  Next, we study the solutions of 
$f_n(\mu, \vx_1) = 0$.  It turns out that these solutions are dramatically different depending on the size of $\mu$.  If $\mu > (1 + \gamma^{-1/2})^2$,
then we can write
\begin{align*}
    \mu 
        &= 
            \mu^0 
            + 
            \frac{1}{\sqrt{n}}
            \mu^1 
            + 
            \ohP\left( \frac{1}{\sqrt{n}} \right), \\
    \vx_1
        &=
            \vx_1^0
            +
            \frac{1}{\sqrt{n}}
            \vx_1^1
            + 
            \ohP\left( \frac{1}{\sqrt{n}} \right).
\end{align*}
Otherwise, the solution is
\begin{align*}
    \mu
        &=
            \mu^0
            + 
            \ohP\left( \frac{1}{\sqrt{n}} \right), \\
    \vx_1
        &=
            \vx_1^0
            +
            \ohP\left( \frac{1}{\sqrt{n}} \right).
\end{align*}
Once we have a solution to the secular equation, we can construct an
eigenvector $\vx$ of $\mS_n^0$ using equation~\eqref{E:x-from-x1}.  Finally, we will use another perturbation argument to express the eigenvalues and eigenvectors of $\mS_n$ in terms of those of $\mS_n^0$ and $\mS_n^1$.

\section{Analysis of the secular equation}

By making use of the Sherman-Morrison-Wordbury formula~\cite{golub1996mc}, we can simplify $f_n(\mu, \vx_1)$.  We first write
\begin{align*}
    \left(
        \mS_{n,22}^0
    \right)
\end{align*}

\clearpage



\section{Linear functionals of eigenvalues}

In the analysis that follows, we will often need to look at sums of the form
\[
    \frac{1}{p} \sum_{\alpha=1}^p g (\lambda_\alpha).
\]
Such sums are called linear functionals of the eigenvalues.  They have been
studied quite extensively for various classes of random matrices.  Hiai and Petz~\cite{hiai1998edw} derived a large deviation principle for the spectral measure  of a Wishart matrix, which implies a strong law of large numbers.  For a  general class of Wishart-like matrices, Bai and Silverstein~\cite{bai2004clt} proved a central limit theorem for linear functionals of eigenvalues.  These two results are much stronger than what we will need.  We summarize the parts of the results that we need in the following two lemmas. For both of them, $g: \reals \to \reals$ is any function analytic on an open interval containing $[a,b]$  and $\bar g = \int g(\lambda) d\FMP_\gamma (\lambda)$.

\begin{lemma}[Hiai \& Petz]\label{L:wishart-lln}
    A large deviations principle holds for the empirical law of 
    $\{ \lambda_\alpha \}$.  Combined with the Borel-Cantelli Lemma, this
    implies that a strong law of large numbers holds for $g(\lambda_\alpha)$. 
    Specifically,
    \[
        \frac{1}{p} \sum_{\alpha=1}^p g( \lambda_\alpha ) \toas \bar g.
    \]
\end{lemma}
    
\begin{lemma}[Bai \& Silverstein]\label{L:wishart-clt}
    A central limit theorem holds for $g(\lambda_\alpha)$.  Although it is
    possible to obtain explicit mean and variance, a crude form of the
    result is that
    \[
        \frac{1}{p} 
        \sum_{\alpha=1}^p
            g( \lambda_\alpha ) - \bar g 
        = O_P(p^{-1}).
    \]  
\end{lemma}

Note that in the previous lemma, the scaling is $p$ instead of $\sqrt{p}$ like in the classical central limit theorem.

It will be convenient to consider functions that are continuous on the support
of $\FMP_\gamma$ but not necessarily on all of $\reals$.  A result of Yin, Bai, and
Krishnaiah \cite{yin1988lle} is that $\lambda_1$ converges almost surely to
$\bar b$.  Further, Bai and Yin \cite{bai1993lse} later proved that
$\lambda_M$ converges almost surely to $\bar a$.  Thus, if $g$ is 
bounded and continuous on $[\bar a, \bar b]$ and we define 
\[
    \tilde g (x)
    =
    \begin{cases}
        g(x), &\text{ when $x \in [\bar a, \bar b]$} \\
        0,    &\text{otherwise,}
    \end{cases}
\]
then $(1/M) \sum_{\alpha=1}^M [ g(\lambda) - \tilde g (\lambda)] \toas 0$.  
Therefore, the above two lemmas hold when $g$ is bounded and continuous only
on $[\bar a, \bar b]$.

We will need to know the limits of certain functionals.  For
$z, z_1, z_2 > \bar b$, we define
\begin{align*}
    \rho_1(z)
        &=
        \frac{1}{2 \gamma}
        -
        \frac{\sqrt{(z - b)(z - a)} + \sqrt{ a b}}
             {2 \gamma z}, \\
    \rho_2(z)
        &= -\rho_{1}'(z) \\
\intertext{and}
    \rho_2(z_1, z_2)
        &= - \frac{r_1(z_2) - r_1(z_1)}{z_2 - z_1}.
\end{align*}

\noindent
When $z_1 \to z_2$ we get that $\rho_2 (z_1, z_2) \to - \rho_1'(z_1)$.  In this way we can extend $\rho_2 (z_1, z_2)$ to the case when $z_1 = z_2$.

Calculations in the Appendix show that
\begin{align}
    r_{1}(z)
        &\equiv \int \frac{d\FMP_\gamma(\lambda)}{z - \lambda} \notag \\
        &= \begin{cases}
               \rho_1(z) &\text{when $\gamma \leq 1$} \\
               \rho_1(z) + (1 - \gamma^{-1}) z^{-1} &\text{otherwise,}
           \end{cases} 
           \label{E:r1-z} \\
    r_{2}(z)
        &\equiv \int \frac{d\FMP_\gamma(\lambda)}{(z - \lambda)^2} \notag \\
        &= \begin{cases}
               \rho_2(z) &\text{when $\gamma \leq 1$} \\
               \rho_2(z) + (1 - \gamma^{-1}) z^{-2} &\text{otherwise,}
           \end{cases} 
           \label{E:r2-z} \\
\intertext{and}
    r_2(z_1, z_2)
        &\equiv \int \frac{d\FMP_\gamma(\lambda)}
                          {(z_1 - \lambda)(z_2 - \lambda)} \notag \\    
        &= \begin{cases}
               \rho_2(z_1, z_2) &\text{when $c \leq 1$} \\
               \rho_2(z_1, z_2) + (1 - \gamma^{-1}) z_1^{-1} z_2^{-1} &\text{otherwise.}
           \end{cases} 
           \label{E:r2-z1-z2}
\end{align}


\section{The resolvent}\label{S:resolvent}

In this section we study the resolvent $R_t(z) = O^T (z I - \Lambda)^{-t} O$ 
and its interaction with the matrices $\Psi_n$ and $U$.  Recall from equation
\eqref{E:psi} that $\Psi_n = V D_n + (1/n^{\half}) E_1^T$.  We have that
$O \Psi_n = O V D_n + (1/n^{\half}) \tilde E_1^T$, where $\tilde E_1 = E_1 O$.
Due the the dependence between the rows of $O V$, it is inconvenient to
deal with $O \Psi_n$ directly.  Let $Z$ be a $p \times k$ matrix of \iid standard normals.  Since $O V$ is uniformly distributed over the Steifel manifold, we have
\begin{align}\label{E:OU}
    O V
        &\overset{d}{=} Z (Z^T Z)^{-T/2}  \notag \\
        &= (1/p^\half) Z \big((1/p) Z^T Z\big)^{-T/2}.  
\end{align}    
We know that
$(1/p) Z^T Z \toas I_k$.  Defining $G = (1/p) Z^T Z - I_k$ and using the
delta method, it is not hard to see that $G = O_P(1/p^\half)$.  For the rest
of the section, we will use the representation in \eqref{E:OU} for $OV$.  We
define $\tilde V = (1/p^\half) Z$,
$\tilde \Psi_n =  \tilde V D_n + (\gamma^{\half} / p^\half) \tilde E_1$,
and $\tilde R_t (z) = (z I_M - \Lambda)^{-t}$.  We also let 
$\tilde R_2(z_1, z_2) = \tilde R_1 (z_1) \tilde R_1 (z_2)$.


\subsection{Almost sure limits}

We first demonstrate that $\tilde \Psi_n^T \tilde R_p(z) \tilde \Psi_n$ and 
$\Psi_n^T R_p (z) \Psi_n$ are asymptotically almost surely
equal. A similar result holds for the pairs 
$(\tilde \Psi_n^T \tilde R_2(z_1, z_2) \tilde \Psi_n,
  \Psi_n^T R_2(z_1, z_2) \Psi_n)$
and $(\tilde V^T \tilde R_1(z) \tilde \Psi_n, V^T R_1(z) \Psi_n)$.
To see this for $\Psi_n^T R_p(z) \Psi_n$, note that
\begin{align*}
    \begin{split}
    \Psi_n^T R_t (z) \Psi_n
    &=
    \tilde \Psi_n^T \tilde R_t(z) \tilde \Psi_n \\
        &+ \frac{1}{p} \Big\{
        G^T D_n^T Z^T \tilde R_p(z) ( Z D_n + \gamma^\half \tilde E_1) \\
        &\qquad \qquad +
        ( Z D_n + \gamma^\half \tilde E_1)^T \tilde R_p(z) Z D_n G \\
        &\qquad \qquad +
        G^T D_n^T Z^T \tilde R_p(z) Z D_n G
        \Big\}.
    \end{split}
\end{align*}
The remainder terms all go to $0$ almost surely.
Proposition~\ref{P:weighted-slln}, a strong law of large numbers for weigted sums,
makes this more rigorous.  The term 
$\tilde \Psi_n^T \tilde R_t(z) \tilde \Psi_n$ is a sum of the form
\[
    \tilde \Psi_n^T \tilde R_t(z) \tilde \Psi_n 
    =
    \sum_{\alpha=1}^p
        \frac{\tilde \vpsi_\alpha \tilde \vpsi^T_\alpha }
             {(z - \lambda_\alpha)^t},
\]
where $\tilde \vpsi_1, \ldots, \tilde \vpsi_p$ are the rows of $\tilde \Psi_n$.
Since the $\tilde \vpsi_\alpha$ are i.i.d, it is quite plausible that the sum
converges to $r_1(z) \E[ \tilde \Psi_n^T \tilde \Psi_n ]$.  Indeed, the strong
law of Proposition~\ref{P:weighted-slln} proves exactly this.  With the additional
help of Lemma~\ref{L:wishart-lln}, we can prove:

\begin{lemma}\label{L:Sbar}
    For $z > \bar b$, 
    \begin{align*}
        S_{n,1}(z) &\toas r_1(z) \left( D^2 + \gamma I_k \right), \\
        S_{n,2}(z) &\toas r_2(z) \left( D^2 + \gamma I_k \right), \\
    \intertext{and}
        T_n(z) &\toas r_1(z) D.
    \end{align*}
\end{lemma}

For convenience, we define
\begin{align*}
    \bar S_1(z) &= r_1(z) \left( D^2 + \gamma I_k \right), \\
    \bar S_2(z) &= r_2(z) \left( D^2 + \gamma I_k \right), \\
\intertext{and}
    \bar T(z) &= r_1(z) D.
\end{align*}

\subsection{Asymptotic normality}

Slightly trickier but still tractable is to get the asymptotic distributions
of $\sqrt{n} \big( S_{n,1}(z) - \bar S_1 (z) \big)$ and the other similar expressions.
Assume that there exist random matrix-valued functions 
$S_1(z)$, $S_2(z)$, and $T(z)$ defined on some probability space
such that for some finite set $\{ z_j \}_{j=1}^J$ with $z_j > b$ for all $j$, 
the matrices
\begin{equation*}
    \left\{ 
        \sqrt{n} \big( S_{n,1}(z_j) - \bar S_1 (z_j) \big), \quad
        \sqrt{n} \big( S_{n,2}(z_j) - \bar S_2 (z_j) \big), \quad
        \sqrt{n} \big( T_n(z_j) - \bar T (z_j) \big) 
    \right\}_{j=1}^{J}
\end{equation*}
converge jointly in distribution to
\[
    \Big\{
        S_1(z_j), S_2(z_j), T(z_j)
    \Big\}_{j=1}^{J}.
\]
The purpose of this section is to prove that such an assumption is justified
and to describe explicitly this limiting distribution. 

We first start with $S_{n,1}$.  We have that
\begin{align*}
    \sqrt{n} \big( S_{n,1} (z) &- \bar S_1 (z) \big) = \\
        \gamma^{-\half} \sqrt{p} &\Bigg\{
            D_n 
            \left( \frac{1}{p} Z^T Z \right)^{-\half}
            \left(
                \frac{1}{p} Z^T (z I_p - \Lambda)^{-1} Z
                -
                r_1(z) I_k
            \right)
            \left( \frac{1}{p} Z^T Z \right)^{-\Thalf}
            D_n \\
            &-{}
            r_1(z)
            D_n 
            \left( \frac{1}{p} Z^T Z \right)^{-\half}
            \left(
                \frac{1}{p} Z^T Z - I_k
            \right)
            \left( \frac{1}{p} Z^T Z \right)^{-\Thalf}
            D_n \\
            &+{}
            r_1(z) \left( D_n^2 - D^2 \right) \\
            &+{}
            \gamma^\half
            D_n 
            \left( \frac{1}{p} Z^T Z \right)^{-\half}
            \left(
                \frac{1}{p} Z^T (z I_p - \Lambda)^{-1} \tilde E_1
            \right) \\
            &+{}
            \gamma^\half
            \left(
                \frac{1}{p} \tilde E_1^T (z I_p - \Lambda)^{-1} Z
            \right)
            \left( \frac{1}{p} Z^T Z \right)^{-\Thalf}
            D_n \\
            &+{}
            \gamma
            \left(
                \frac{1}{p} \tilde E_1^T (z I_p - \Lambda)^{-1} \tilde E_1
                -
                r_1(z) I_k
            \right)
        \Bigg\}.
\end{align*}
Employing Slutzky's theorem, this has the same limiting distribution as
\begin{align*}
    S_{n,1}^{(1)}(z) \equiv 
        \gamma^{-\half} \sqrt{p} &\Bigg\{
            D
            \left(
                \frac{1}{p} Z^T (z I_p - \Lambda)^{-1} Z
                -
                r_1(z) I_k
            \right)
            D \\
            &-{}
            r_1(z)
            D
            \left(
                \frac{1}{p} Z^T Z - I_k
            \right)
            D \\
            &+{}
            r_1(z) \left( D_n^2 - D^2 \right) \\
            &+{}
            \gamma^\half
            D 
            \left(
                \frac{1}{p} Z^T (z I_p - \Lambda)^{-1} \tilde E_1
            \right) \\
            &+{}
            \gamma^\half
            \left(
                \frac{1}{p} \tilde E_1^T (z I_p - \Lambda)^{-1} Z
            \right)
            D \\
            &+{}
            \gamma
            \left(
                \frac{1}{p} \tilde E_1^T (z I_p - \Lambda)^{-1} \tilde E_1
                -
                r_1(z) I_k
            \right)
        \Bigg\}.
\end{align*}
Combining Lemmas \ref{L:wishart-lln}, \ref{L:wishart-clt}, and Proposition~\ref{P:weighted-clt}, we can see that $S_1(z)$, the limiting distribution of
$S_{n,1}^{(1)}(z)$ has elements above and along the main diagonal converging
to a multivariate normal with
\begin{align}
    \cov[ (S_1(z_1))_{ij}, (S_1(z_2))_{ij} ]
    &= \gamma^{-1} (1 + \delta_{ij}) \mu_i \mu_j 
                 \big( r_2 (z_1, z_2) - r_1(z_1) r_1 (z_2) \big) \notag \\
      &+ (1 + \delta_{ij}) (\mu_i + \mu_j + \gamma) r_2(z_1, z_2) \notag \\
      &+ \delta_{ij} \sigma_{ii} r_1(z_1) r_1(z_2), \label{E:cov-S1-1}
\end{align}
and for $i \neq j$,
\begin{align}
    \cov[ (S_1(z_1))_{ii}, (S_1(z_2))_{jj} ]
    &= \sigma_{ij} r_1(z_1) r_1(z_2). \label{E:cov-S1-2}
\end{align}
All other correlations are $0$.

\section{Limits of singular values}

We are finally ready to study the limits of the top eigenvalues of
$(1/n) X_n^T X_n$.  We have that
$(1/n) X_n^T X_n = \Psi_n \Psi_n^T + O \Lambda O^T$. Recall from
Lemma~\ref{L:secular} that the $j$th eigenvalue of $(1/n) X_n^T X_n$, say 
$\hat \mu_j$, is either equal to some $\lambda_i$ or else the matrix
$S_{n,1}(\hat \mu_j)$ has a unit eigenvalue; conversely all values of $z$ with
$S_{n,1}(z)$ having a unit eigenvalue are eigenvalues of $(1/n) X_n^T X_n$.
Lemma~\ref{L:lowrank-perturb}, in the Appendix, shows that  almost surely
$(1/n) X_n^T X_n$ has no nonzero eigenvalue equal to any $\lambda_i$.  Thus,
we can study the nonzero eigenvalues of $(1/n) X_n^T X_n$ by studying the
random matrix-valued function $S_{n,1}(z)$.

We denote by $\hat \mu_i$ the $i$th largest eigenvalue of $(1/n) X_n^T X_n$.

\subsection{Almost sure limits}\label{SS:value-limit}

Assume for now that $\gamma \leq 1$.  Denote by $\bar k$ the maximum index such that $\mu_{\bar k} > \sqrt{\gamma}$.  For $\varepsilon$ small enough, there are exactly $\bar k$ values of $z$ in  $(b + \varepsilon, \infty)$ for which $\bar S_1(z)$, the limit of $S_{n,1}(z)$ has a unit eigenvalue.  This is because 
\(
    \bar S_1(z) 
    = 
    \diag( r_1(z) (\mu_1 + \gamma), \ldots, r_1(z) (\mu_K + \gamma))
\)
and $r_1(z) < (\sqrt{\gamma} + \gamma)^{-1}$ on $(b, \infty)$.  A straightforward calculation shows that $r_1(z)$ has inverse
\[
    z(r_1)
    =
    \frac{1}{r_1}
    + 
    \frac{1}{1 - \gamma r_1}
\]
For $i \leq \bar K$, we define
\begin{align*}
    \tilde \mu_i
    &\equiv z\left( (\mu_i + \gamma)^{-1} \right) \\
    &= \left( \mu_i + 1 \right) 
       \left( 1 + \frac{\gamma}{\mu_i} \right)
\end{align*}
Due to the continuity of eigenvalues, for $i$ in this range, $\tilde \mu_i$ is
almost surely a limiting eigenvalue of $(1/n) X_n^T X_n$.  Thus for $i \leq K$,
$\hat \mu_i \toas \tilde \mu_i$.

Since $(1/n) O^T X_n^T X_n O - \Lambda$ is positive-semidefinite, the
Courant-Fischer min-max characterization of eigenvalues tells us that
$\hat \mu_i \geq \lambda_i$. Therefore, for $\bar k < i \leq k$,
\[
    b
    \leq
    \liminf_{n \to \infty} \hat \mu_i
    \leq
    \limsup_{n \to \infty} \hat \mu_i    
    \leq
    b + \varepsilon.
\]
Defining $\tilde \mu_i = b$ for $\bar k < i \leq k$, we can let
$\varepsilon \to 0$ to get that for $i$ in this range,
$\hat \mu_i \toas \tilde \mu_i$.

Now suppose that $\gamma > 1$.  We can use the work above to derive the limits of the top singular values since the nontrivial eigenvalues of $(1/n) X^T X$ and $\gamma (1/p) X X^T$ are the same.  If we replace $\gamma$ by $\gamma^{-1}$ and $\mu$ by $\mu \gamma^{-1}$, then we can apply the above theory.  The critical threshold is $\mu_i \gamma^{-1} > \gamma^{-1/2}$, i.e. $\mu_i > \sqrt{\gamma}$.  Above this threshold, the limit is given by $\gamma ( \mu \gamma^{-1} + 1)(1 + 1/\mu) = (\mu + 1)\left( 1 + \gamma/\mu \right)$.  Below the threshold, the limit is $\gamma (1 + \gamma^{-1/2})^2 = (1 + \gamma^{1/2})^2$.  We see that the formulas are the same as for $\gamma \leq 1$.

\subsection{Asymptotic normality}

Next, we turn our attention to the asymptotic distribution of 
$\sqrt{n} (\hat \mu_i - \tilde \mu_i)$.  Again, we start by assuming that $\gamma \leq 1$.  Let $\nu_i(z)$ denote the $i$th
eigenvalue of $S_{n,1}(z)$.  Since the off-diagonal elements of $S_{n,1}(z)$ are
of size $O_P(1/n^\half)$ and the diagonal elements are of size
$O(1) + O_P(1/n^\half)$, a standard perturbation argument
(which can be found, for example in \cite{anderson1963atp}) shows that
$\nu_i(z) = (S_{n,1}(z))_{ii} + O_P(1/n)$.
Since $S_{n,1}(z) \eqd \bar S_{n,1}(z) + (1/n^\half) S_1(z) + O_P(1/n)$, we
have that 
$\nu_i(z) \eqd r_1(z) (\mu_i + \gamma) + (1/n^\half) (S_1(z))_{ii} + O_P(1/n)$.
For $i \geq \bar k$, using the Taylor expansion 
$r_1(z) = r_1 (\tilde \mu_i) + r_1'(\tilde \mu_i) (z - \tilde \mu_i) + O(|z - \tilde \mu_i|^2)$
and the fact that $\nu_i (\hat \mu_i) = 1$, we have that
\[
    1 \eqd 1 
            + r_1' (\tilde \mu_i)(\hat \mu_i - \tilde \mu_i)(\mu_i + \gamma) 
            + (1/n^\half) (S_1(\hat \mu_i))_{ii}
            + O(|\hat \mu_i - \tilde \mu_i|^2) 
            + O_P(1/n)
\]
so that
\[
    \sqrt{n} (\hat \mu_i - \tilde \mu_i)
    \eqd
        -\frac{(S_1(\hat \mu_i))_{ii}}
              {r_1' (\tilde \mu_i) (\mu_i + \gamma)}
        + O(\sqrt{n}|\hat \mu_i - \tilde \mu_i|^2)       
        + O_P(1/n^\half).
\]
This in turn implies that $(\hat \mu_i - \tilde \mu_i) = O_P(1/n^\half)$.
Using a Taylor expansion of $S_1(z)$ about $\tilde \mu_i$, we get that jointly
for $i \leq \bar k$,
\[
    \sqrt{n} (\hat \mu_i - \tilde \mu_i)
        \tod -\frac{(S_1(\tilde \mu_i))_{ii}}
                   {r_1' (\tilde \mu_i) (\mu_i + \gamma)}.
\]
It is straightforward to show that 
\(
    r_2(\tilde \mu_i, \tilde \mu_j) 
    =
    \mu_i \mu_j 
    / 
    [ (\mu_i \mu_j - \gamma) (\mu_i + \gamma) (\mu_j + \gamma) ].
\)
Putting $i = j$, we have that
\(
    \left[ -r_1'(\tilde \mu_i ) \right]^{-1}
    =
    \left[ r_2(\tilde \mu_i, \tilde \mu_i) \right]^{-1}
    =
    \left( \mu_i + \gamma \right)^2
    \left( 1 - \gamma/\mu_i^2 \right)
\)
The previous section shows that jointly the unique elements
of $\{ S_1(\tilde \mu_i) \}_{i=1}^{\bar K}$ have a multivariate normal distribution.
Equations~\eqref{E:cov-S1-1}~and~\eqref{E:cov-S1-2} give
\begin{align*}
    \var [ (S_1(\tilde \mu_i))_{ii} ]
        &= \frac{1}{(\mu_i + \gamma)^2}
           \left[
               2 ( 2 \mu_i + \gamma + 1 )
               \left(
                   \frac{\mu_i^2}{\mu_i^2 - \gamma}
               \right)
               +
               \sigma_{ii}
           \right] \\
\intertext{and for $i \neq j$}
    \cov [ (S_1(\tilde \mu_i))_{ii},  (S_1(\tilde \mu_j))_{jj}]
        &= \frac{\sigma_{ij}}
                {(\mu_i + \gamma)(\mu_j + \gamma)}.
\end{align*}
Thus, the vector 
\(
    \sqrt{n} 
    (\hat \mu_1 - \tilde \mu_1, 
     \ldots, 
     \hat \mu_{\bar k} - \tilde \mu_{\bar k}
    )
\)
converges in distribution to a multivariate normal with the covariance
between elements $i$ and $j$ given by
\[
    \tilde \sigma_{ij}
    \equiv
    \sigma_{ij} 
    \frac{(\mu_i^2 - \gamma)(\mu_j^2 - \gamma)}{\mu_i^2 \mu_j^2}
    +
    \delta_{ij}
    2(2\mu_i + \gamma + 1)
    \left(
        \frac{\mu_i^2 - \gamma}{\mu_i^2}
    \right).
\]

When $\bar k < \mu_i \leq k$, to get asymptotic variance of
$\sqrt{n} (\hat \mu_i - \tilde \mu_i)$.  We must know finer asymptotics of
the leading eigenvalues $\lambda_1, \ldots, \lambda_k$.  Johnstone
\cite{johnstone2001dle} showed that $\lambda_1 = b + O_P(1/n^\frac{2}{3})$.
Soshnikov \cite{soshnikov2002nud} generalized this result to the top
$k$ eigenvalues, for fixed $k$.  Let $\varepsilon_n = \lambda_1 - b$  and
$\varepsilon_n' = b - \lambda_i$.  Denote by $Z_n$ the random variable
$b + 2 \varepsilon_n$. Since $Z_n > \lambda_1$ and $Z_n \toas b$,
the smallest eigenvalue of $S_{n,1} (Z_n)$ converges almost surely to a
value greater than $1$.  Thus, 
$\limsup_{n\to\infty} \hat \mu_i \leq \liminf Z_n$, and further
\[
    \limsup_{n\to\infty} \sqrt{n} \varepsilon_n'
    \leq
    \liminf_{n\to\infty} \sqrt{n} (\hat \mu_i - b)
    \leq
    \limsup_{n\to\infty} \sqrt{n} (\hat \mu_i - b)
    \leq
    \liminf_{n\to\infty} \sqrt{n} (2 \varepsilon_n).
\]
Therefore, $\sqrt{n} (\hat \mu_i - b) \toP 0$.

Now suppose that $\gamma > 1$.  Using the same transformation as the one at the end of Subsection~\ref{SS:value-limit}, it is straightforward to 
check that we get the same expression for the asymptotic covariance.

\section{Limits of singular vectors}

In this section we derive the limiting behavior of the singular vectors
of $(1/n) X_n^T X_n$, looking at the limits of $V^T \hat V$.

First suppose that $i \leq \bar k$.  Then $\hat \mu_i \toas \tilde \mu_i$ and
$S_{n,1}(\hat \mu_i) \toas \bar S_1 (\tilde \mu_i)$.  The $i$th eigenvalue
of $S_{n,1} (\hat \mu_i)$ converges almost surely to $1$ and the $i$th
eigenvector converges almost surely to $\be_i$.  This fact, 
Lemmas~\ref{L:secular}, Lemma~\ref{L:Sbar}, and equation~\eqref{E:normY} 
tell us that $\hat \bv_i$, the normalized $i$th eigenvector of
$(1/n) X_n^T X_n$ almost surely has the same limit as
\[
    (\bar S_2(\tilde \mu_i))_{ii}^{-\half} R_1(\tilde \mu_i) \Psi_n \be_i.
\]
Combining this result with equation~\eqref{E:corrY}, we get that
\begin{align*}
    V^T \hat \bv_i
        &\toas
        (\bar S_2(\tilde \mu_i))_{ii}^{-\half} \bar T(\tilde \mu_i) \be_i \\
        &= \left( r_2( \tilde \mu_i ) (\mu_i + \gamma )\right)^\half
           \left( r_1( \tilde \mu_i ) \mu_i^\half \right) \\
        &= \sqrt{
                \frac{ 1 - \gamma/\mu_i^2}
                     { 1 + \gamma/\mu_i  }
           }
\end{align*}

Next say that $\bar k < i \leq k$.  In this case, since 
$\lim_{z \to b^+} r_2(z) = \infty$ and $\lim_{z \to b^+} r_1(z) < \infty$, we must have that  $V^T \hat v_i \toas 0$.

\section*{Appendix A: Integral calculations}

Here we compute the limits of certain linear functionals of Wishart
eigenvalues.

\begin{lemma}
    For $z > b$, and $\gamma > 0$, $\int \frac{\fMP_\gamma(x)}{z - x} dx$
    is equal to
    \[
        \rho_1(z) 
        =
        \frac{1}{2 \gamma}
        -
        \frac{ \sqrt{(z - b)(z - a)} + \sqrt{ a b } }
             { 2 \gamma z}.
    \]
\end{lemma}
\begin{proof}
    To compute the integral, we make the substitution 
    $x = 2 \gamma^{1/2} t + 1 + \gamma$ and then set $\cos \theta = t$.
    Using the identity that for $|y| > 1$,
    \(
        \int_0^\pi \frac{d\theta}{\cos \theta + y}
        = \sgn(y) \frac{\pi}{(y^2 - 1)^\half},
    \)
    the result follows.
\end{proof}

\begin{lemma}
    For $z > b$ and $\gamma > 0$, the integral
    $\rho_2(z) = \int \frac{\fMP_\gamma(x)}{(z - x)^2} dx$ is equal to
    \[
        -\rho_{1}'(z)
        =
        \frac{1}{2 \gamma z^2}
        \left(
            \frac{(a + b) z - 2 a b}
                 {2 \sqrt{(z - b)(z - a)}}
            -
            \sqrt{a b}
        \right).
    \]
\end{lemma}
\begin{proof}
    From the previous lemma we have that
    $\rho_{1}(z) = \int \frac{\fMP_\gamma(x)}{z - x} dx$.
    We can differentiate under the integral to get that
    $\int \frac{\fMP_\gamma(x)}{(z - x)^2} dx = - \rho_{1}'(z)$.    
\end{proof}

\begin{lemma}
For $z_1, z_2 > \bar b$, $\gamma > 0$, and $z_1 \neq z_2$, the integral
    $\int \frac{\fMP_\gamma(x)}{(z_1 - x)(z_2 - x)} dx$ is equal to
    \[
        \rho_2(z_1, z_2)
        =
        -
        \frac{\rho_1(z_2) - \rho_1(z_1)}{z_2 - z_1}.
    \]
\end{lemma}
\begin{proof}
    We have that
    \[
        \rho_2 (z_1, z_2)
        =
        \frac{1}{2 \pi \gamma}
        \int_{a}^{b}
        \frac{\sqrt{(x - a)(b - x)}}
             {x (z_1 - x) (z_2 - x)}
        dx.
    \]
    We first make the substitution 
    $x = \frac{b - a}{2} t + \frac{b + a}{2} = 2 \gamma^{1/2} t + 1 + \gamma$.
    Letting $\beta = (\gamma^{1/2} + \gamma^{-1/2})/2$ and
    $\alpha_i = -\frac{z_i}{2 \gamma^{1/2}} + \beta$, we have
    \begin{align*}
        \rho_2(z_1, z_2)
        &=
        \frac{1}{2 \pi \gamma}
        \int_{-1}^{1}
        \frac{4 \gamma \sqrt{(t + 1)(1 - t)}}
             {(2 \gamma^{1/2} t + 1 + \gamma)
              (z_1 - 2 \gamma^{1/2} t - 1 - \gamma)
              (z_2 - 2 \gamma^{1/2} t - 1 - \gamma)}
        dt \\
        &=
        \frac{1}{4 \pi \gamma^{3/2}}
        \int_{-1}^{1}
        \frac{\sqrt{1 - t^2}}
             {(t + \alpha_1)(t + \alpha_2)(t + \beta)}
        dt.
    \end{align*}
    We make the substitution $t = \cos \theta$ to get
    \[
        \rho_2(z_1, z_2)
        =
        \frac{1}{4 \pi \gamma^{3/2}}
        \int_0^\pi
        \frac{1 - \cos^2 \theta}
             {(\cos \theta + \alpha_1)
              (\cos \theta + \alpha_2)
              (\cos \theta + \beta)}
        d\theta.
    \]
    We expand the fraction as
    \[
        \frac{1 - \cos^2 \theta}
             {(\cos \theta + \alpha_1)
              (\cos \theta + \alpha_2)
              (\cos \theta + \beta)}
        =
        \frac{A_1}{\cos \theta + \alpha_1}
        +
        \frac{A_2}{\cos \theta + \alpha_2}
        +
        \frac{B}{\cos \theta + \beta}.
    \]
    This gives us the system of equations
    \[
        \left(
        \begin{matrix}
            1                  & 1                  & 1                \\
            (\alpha_2 + \beta) & (\alpha_1 + \beta) & (\alpha_1 + \alpha_2) \\
            \alpha_2 \beta     & \alpha_1 \beta     & \alpha_1 \alpha_2
        \end{matrix}
        \right)
        \left(
        \begin{matrix}
            A_1 \\
            A_2 \\
            B
        \end{matrix}
        \right)
        =
        \left(
        \begin{matrix}
            -1 \\
             0 \\
             1
        \end{matrix}
        \right).
    \]
    We employ Cramer's rule to get that
    \[
        A_1
        =
        \frac{(\alpha_1^2 - 1)(\beta - \alpha_2)}
             {\alpha_1^2 (\alpha_2 - \beta)
              + \alpha_2^2  (\beta - \alpha)
              + \beta^2 (\alpha_1 - \alpha_2)},
    \]
    with a similar expression for $A_2$ and $B$.  Using the identity that
    for $|y| > 1$,
    \(
        \int_0^\pi \frac{d\theta}{\cos \theta + y}
        = \sgn(y) \frac{\pi}{(y^2 - 1)^\half},
    \)
    and noting that $\alpha_1$ and $\alpha_2$ are both negative while $\beta$ 
    is positive, we get that the integral evaluates to
    \begin{align*}
        \rho_2 (z_1, z_2)
        &=
        \frac{1}{4 \gamma^{3/2}}
        \cdot
            \frac{(\alpha_1^2 - 1)^\half (\alpha_2 - \beta)
                   + (\alpha_2^2 - 1)^\half (\beta - \alpha_1)
                   + (\beta^2 - 1)^\half (\alpha_2 - \alpha_1)}
                 {\alpha_1^2 (\alpha_2 - \beta)
                   + \alpha_2^2  (\beta - \alpha_1)
                   + \beta^2 (\alpha_1 - \alpha_2)}.
    \end{align*}
    We can simplify the above expression using that 
    $\alpha_1 - \alpha_2 = (z_2 - z_1)/(2 \gamma^{1/2})$, 
    $\beta - \alpha_1 = z_1/(2 \gamma^{1/2})$, and
    $\alpha_2 - \beta = -z_2/(2 \gamma^{1/2})$ as
    \[
        \rho_2 (z_1, z_2)
        =
        \frac{1}{4 \gamma^{3/2}}
        \cdot
            \frac{- z_2 (\alpha_1^2 - 1)^\half
                   + z_1 (\alpha_2^2 - 1)^\half 
                   + (z_1 - z_2)(\beta^2 - 1)^\half }
                 {-z_2 \alpha_1^2 
                   + z_1 \alpha_2^2 
                   + (z_2 - z_1) \beta^2 }.    
    \]
    After some algebra, we see that the expression simplifies to
    \[
        \rho_2 (z_1, z_2)
        =
        \frac{ z_1 \sqrt{ (z_2 - a) (z_2 - b)}
               - z_2 \sqrt{ (z_1 - a) (z_1 - b)}
               + (z_1 - z_2) \sqrt{ a b} }
             {2 \gamma z_1 z_2 (z_2 - z_1)}.
    \]
\end{proof}


\section*{Appendix B: Low-rank perturbations of diagonal matrices}

Our aim in this section is to prove the following lemma:

\begin{lemma}\label{L:lowrank-perturb}
    Let $A$ be a random $n\times n$ symmetric matrix with rank $k$, and
    $0 < k < n$.  Suppose that $A$ is orthogonally-invariant, in the sense that 
    $O^T A O \eqd A$ for all orthogonal $O$ independent of $A$.  Let
    $D = \diag(d_1, d_2, \ldots, d_n)$ be a diagonal matrix.  Then, if
    $\#\{ i : d_i = d \} \leq k$, then almost surely $A + D$ has no eigenvalue
    equal to $d$.
\end{lemma}

\noindent Do to so, we will need two lemmas, which we state and prove now.

\begin{lemma}
    Let $A$ be as above.  Partition $A$ as 
    \[
        A
        =
        \left(
        \begin{matrix}
            a_{11}   & \ba_{12}^T \\
            \ba_{21} & A_{22}
        \end{matrix}
        \right)
    \]
    and define $A_2 = A_{22} - a_{11}^{-1} \ba_{21} \ba_{12}^T$.  Then:
    \begin{enumerate}[(i)]
        \item $a_{11}$ has a continuous density,
        \item $A_2$ almost surely has rank $k-1$,
        \item $A_2$ is orthogonally-invariant.
    \end{enumerate}
\end{lemma}
\begin{proof}
    The matrix $A$ is equal in distribution to $U \Lambda U^T$ for a
    Haar-distributed $U$ and some diagonal $\Lambda$ independent of $U$.  
    Without loss of generality, say the nonzero entries of $\Lambda$ are
    $\lambda_{1}, \ldots, \lambda_k$.  Then 
    $a_{11} \eqd \sum_{i=1}^k \lambda_i (U)_{1i}^2$.  Since $0 < k < n$, 
    and $((U)_{11}, \ldots, (U)_{1(n-1)})$ has a continuous density, this is a
    random variable with continuous density as well.
    
    Since $a_{11}$ has a continuous density, $a_{11} \neq 0$ almost surely.
    Since $A$ can be transformed to
    \(
        \left(
        \begin{matrix}
            a_{11} & \ba_{12}^T \\
            0      & A_2
        \end{matrix}
        \right)
    \)
    via row operations, $A_2$ must have rank one less than $A$.
    
    Letting $O_2$ be any $(n-1)\times(n-1)$ orthogonal matrix, we have
    that
    \[
        \left(
        \begin{matrix}
            1 & 0 \\
            0 & O_2^T
        \end{matrix}
        \right)
        A
        \left(
        \begin{matrix}
            1 & 0 \\
            0 & O_2
        \end{matrix}
        \right)
        =
        \left(
        \begin{matrix}
            a_{11}   & \ba_{12}^T O_2 \\
            O_2^T \ba_{21} & O_2^T A_{22} O_2
        \end{matrix}
        \right).
    \]
    Since $A$ is orthogonally invariant, this implies that $A_2$ is
    as well.
\end{proof}

\begin{lemma}
    Let $A$ be as above except with rank $k$ possibly equal to $0$.  
    Let $D = \diag(d_1, d_2, \ldots, d_n)$ be a diagonal matrix independent of
    $A$ with rank greater than or equal to $n-k$ almost surely.  Then
    $A + D$ almost surely has full rank.
\end{lemma}
\begin{proof}
    If $k = 0$, the result is trivial.  Otherwise,
    without loss of generality, assume that either $D$ is of full rank
    or $d_1 = 0$.  Partition $A$ as in the previous lemma and let 
    \(
        D_2 = \diag(d_2, \ldots, d_n).
    \)
    The determinant of $A + D$ is equal to
    \begin{align*}
        \det ( A + D )
            &= \left|
               \begin{matrix}
                   a_{11} + d_1 & \ba_{12}^T \\
                   \ba_{21}     & A_{22} + D_2
               \end{matrix}
               \right| \\
            &= \left|
               \begin{matrix}
                   a_{11} + d_1 & \ba_{12}^T \\
                   0            & A_{22} - a_{11}^{-1} \ba_{21} \ba_{12}^T + D_2
               \end{matrix}
               \right| \\
            &= (a_{11} + d_1) \det (A_2 + D_2),
    \end{align*}
    We note that $D_2$ is of rank at least $(n-1) - (k-1)$. 
    By the previous lemma, $A_2$ has rank $k - 1 < n - 1$ and is
    orthogonally-invariant.  Since $a_{11}$ has a continuous density and
    is independent of $d_{1}$, $a_{11} + d_1 \neq 0$.  By induction,
    $\det (A_2 + D_2 \neq 0)$ and the result follows.
\end{proof}

Now, Lemma~\ref{L:lowrank-perturb} is an easy corollary:

\begin{proof}[Proof of Lemma~\ref{L:lowrank-perturb}]
    Say $A + D$ has an eigenvalue equal to $d$, with
    $\#\{ i : d_i = d \} \leq k$.  Then $A + D - d I_n$ is not
    of full rank.  Since $D - d I_n$ has at least $n-k$ nonzero entries,
    this contradicts the previous lemma.
\end{proof}    




%\section*{Acknowledgements}

%I would like to thank Persi Diaconis, Zongming Ma, Alexei Onatski, and
% Debashis Paul for their helpful communications and discussions. I would
% also like to thank Art Owen, my thesis advisor, for his support and 
% encouragement.
