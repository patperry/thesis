
\chapter[Behavior of the SVD]{Behavior of the SVD in low-rank plus noise models}


\newcommand{\ba}{\boldsymbol{a}}      
\newcommand{\be}{\boldsymbol{e}}      
\newcommand{\bo}{\boldsymbol{o}}      
\newcommand{\bu}{\boldsymbol{u}}      
\newcommand{\bv}{\boldsymbol{v}}
\newcommand{\by}{\boldsymbol{y}}
\newcommand{\bz}{\boldsymbol{z}}

\newcommand{\tPsi}{\tilde \Psi}       % tilde symbols

\newcommand{\Q}{\mathcal{Q}}
\newcommand{\U}{\mathcal{U}}          % script symbols
\newcommand{\V}{\mathcal{V}}
\newcommand{\W}{\mathcal{W}}


\newcommand{\Thalf}{\frac{T}{2}}
\newcommand{\fourth}{\frac{1}{4}}


\section{Introduction}

Many modern data sets involve simultaneous measurements of a large number of
variables. Some financial applications including portfolio selection involve
looking at the market prices of hundreds or thousands of stocks and their
evolution over time \cite{markowitz1952ps}. Microarray studies
\cite{biotechnol1995sms} involve measuring the expression levels of thousands of
genes simultaneously. Text processing \cite{manning1999fsn} involves counting
the appearances of thousands of words on thousands of documents. In all of these
applications, it is natural to suppose that even though the data are
high-dimensional, their dynamics are driven by a relatively small number of
latent factors.

Under the hypothesis that one ore more latent factors explain the behavior
of a data set, principal component analysis (PCA) \cite{jolliffe2002pca} is a
popular method for estimating these latent factors.  When the dimensionality
of the data is small relative to the sample size, Anderson's 1963 paper
\cite{anderson1963atp} gives a complete treatment of how the procedure
behaves.  Unfortunately, his results do not apply when the sample size is
comparable to the dimensionality.

A further complication with many modern data sets is that it is no longer
appropriate to assume the observations are \iid Also, sometimes it is
difficult to distinguish between ``observation'' and ``variable''. We call such
data ``transposable''. A microarray study involving measurements of $p$ genes
for $n$ different patients can be considered transposable: we can either treat
each gene as a measurement of the patient, or we can treat each patient as a
measurement of the gene. There are correlations both between the genes
\emph{and} between the patients, so in fact both interpretations are relevant
\cite{efron2008smi}.

One can study latent factor models in a transpose-agnostic way be considering
generative models of the form $\mX = \mU \mD \mV^\trans + \mE$. Here, $\mX$ is
the $n \times p$ observed data matrix. The unobserved row and column factors
are given by $\mU$ and $\mV$, respectively, matrices with $k$ orthonormal
columns each, with $k \ll \min(n,p)$. The strengths of the factors are given
in the $k\times k$ diagonal matrix $\mD$, and $\mE$ is a matrix of noise. A
natural estimator for the latent factor term $\mU \mD \mV^\trans$ can be
gotten from truncating the singular value decomposition (SVD)
\cite{golub1996mc} of $\mX$. The goal of this paper is to study the behavior
of the SVD when $n$ and $p$ both tend to infinity, with their ratio tending to
a nonzero constant.

In an upcoming paper, Onatski~\cite{onatski} gives a very thorough treatment
of these models. He assumes that the elements of $E$ are \iid Gaussians, that
$\sqrt{n} ( \mU^\trans \mU - \mI_k) $ tends to a multivariate Gaussian
distribution, and that $\mV$ and $\mD$ are both nonrandom.

Slight variations of the above model have been analyzed by
Baik~\&~Silverstein~\cite{baik2006els}, and by Paul~\cite{paul2007ase}. These
authors look at matrices of the form $\mX = \mT^{1/2} \mZ$, where $\mZ$ is a
matrix with iid entries, and $\mT$ is a ``spiked'' covariance matrix, i.e. a
positive-definite matrix with a few large eigenvalues. Baik \& Silverstein
prove quite general results using the Stieltjes transform only assuming finite
fourth moments, but they limit their study to the almost sure limits of the
top singular values of $\mX$. Paul gives the limiting distributions of the
sample singular values as well as the singular vectors, but he assumes that
the elements of $\mZ$ are independent standard Gaussians. Later, Chen and
coauthors \cite{chen2009ppc} generalized his results to allow for correlated
Gaussian loadings of the factors.

The contributions of this chapter are twofold. First, we work under a
transpose-agnostic generative model that allows randomness in all three of
$\mU$, $\mD$, and $\mV$. Second, we give a more complete picture of the
almost-sure limits of the sample singular vectors, taking into account their
projections onto the space orthogonal to the population singular vectors.

We describe the results in Section 2. The rest of the paper is dedicated to
proving the two main theorems. We owe a substantial debt to Onatski's work.
Although some significant details below are different, the general outline and
the main points of the argument are the same.

\section{Assumptions, notation, and main results}

Here we make explicit what the model and assumptions are, and we present our
main results.

\subsection{Assumptions and notation}

We will work sequences of matrices indexed by $n$, with
\[
    \mX_n = \sqrt{n} \mU_n \mD_n \mV_n^\trans + \mE_n.
\]
We denote by $\sqrt{n} \mU_n \mD_n \mV_n^\trans$ the ``signal'' part of the
matrix $\mX_n$ and $\mE_n$ the ``noise'' part. We will often refer to $\mU_n$
and $\mV_n$ as the left and right factors of $\mX_n$, and the matrix $\mD_n$
will be called the matrix of normalized factor strengths. The first two
assumptions concern the signal part:

\begin{assumption}\label{A:factors}
    The factors $\mU_n$ and $\mV_n$ are random matrices of dimensions
    $n \times k$ and $p \times k$, respectively.  The number of factors, $k$,
    is a fixed constant.  The factors are normalized so that
    $\mU_n^\trans \mU_n =  \mV_n^\trans \mV_n = \mI_k$, where $\mI_k$ denotes 
    the $k \times k$ identity matrix.  The ratio $p/n = \gamma$ is a fixed 
    nonzero scalar in the range $(0,\infty)$.
\end{assumption}
\noindent

\begin{assumption}\label{A:sizes}
    The matrix of factor strengths, $\mD_n$, is of size $k\times k$ and 
    diagonal, with 
    \(
        \mD_n = \diag \left( d_{n,1}, d_{n,2}, \ldots, d_{n,k} \right)
    \)
    and
    $d_{n,1} > d_{n,2} > \cdots > d_{n,k} > 0$.  The matrix $\mD_n$ converges
    almost surely to a deterministic matrix 
    $\mD = \diag( \mu_1^{1/2}, \mu_2^{1/2}, \ldots, \mu_k^{1/2})$ with
    $\mu_1 > \mu_2 > \cdots > \mu_k > 0$. Moreover, the vector
    \(
        \sqrt{n} ( d_{n,1}^2 - \mu_1, d_{n,1}^2 - \mu_2, 
                   \ldots, 
                   d_{n,K}^2 - \mu_k )
    \)
    converges in distribution to a mean-zero multivariate normal with 
    covariance matrix $\mSigma$ having entries $\Sigma_{ij} = \sigma_{ij}$
    (possibly degenerate).
\end{assumption}
\noindent
The next assumption concerns the noise part:

\begin{assumption}\label{A:noise}
    The noise matrix $\mE_n$ is an $n\times p$ matrix with entries 
    $E_{n,ij}$ independent $\Normal(0,\, \sigma^2)$ random variables, also
    independent of $\mU_n$, $\mD_n$, and $\mV_n$.
\end{assumption}
\noindent

We need to introduce some more notation. We denote the columns of $\mU_n$ and
$\mV_n$ by $\vU_{n,1}, \vU_{n,2}, \ldots, \vU_{n,k}$ and $\vV_{n,1},
\vV_{n,2}, \ldots, \vV_{n,k}$, respectively. We let 
\[
    \mX_n \approx \sqrt{n} \mhU_n \mhD_n \mhV_n^\trans
\]
be the singular value decomposition of
$\mX_n$ truncated to $k$ terms, where $\mhD_n = \diag( \hat \mu_{n,1}^{1/2},
\hat \mu_{n,2}^{1/2}, \ldots, \hat \mu_{n,k}^{1/2})$ and the columns of $\mhU$ and $\mhV$ are given
by $\vhU_{n,1}, \vhU_{n,2} \ldots, \vhU_{n,k}$ and $\vhV_{n,1}, \vhV_{n,2},
\ldots, \vhV_{n,k}$, respectively.


\subsection{Main results}

We are now in a position to say what the results are.  There are two main
theorems, one concerning the sample singular values and the other concerning
the sample singular vectors.  First we give the result about the singular
values.
\begin{theorem}\label{T:values}
    Under Assumptions~\ref{A:factors}~--~\ref{A:noise},
    the vector $\hat \vmu = (\hat \mu_1, \hat \mu_2, \ldots, \hat \mu_k)$
    converges almost surely to 
    $\tilde \vmu = (\tilde \mu_1, \tilde \mu_2, \ldots, \tilde \mu_k)$,
    where
    \[
        \tilde \mu_k
        =
        \begin{cases}
            \left( \mu_i + \sigma^2 \right)
            \left( 1 + \gamma \sigma^2 / \mu_i \right)
                &\text{when $\mu_i > \sqrt{\gamma} \sigma^2$}, \\
            \sigma^2 \left( 1 + \sqrt{\gamma} \right)^2
                &\text{otherwise.}
        \end{cases}
    \]
    Moveover, 
    \(
        \sqrt{n} (\hat \vmu - \tilde \vmu)
    \)
    converges in distribution to a (possibly degenerate) multivariate normal
    with covariance matrix $\mtSigma$ whose $ij$ element is given by
    \[
        \tilde \sigma_{ij}
        \equiv
        \begin{cases}
            \begin{aligned}
                \sigma_{ij}
                &\left(
                    1 - \gamma \sigma^4 / \mu_i^2
                \right)
                \left(
                    1 - \gamma \sigma^4 / \mu_j^2
                \right)
                \\
                &+ \delta_{ij}
                2
                \sigma^2
                \left(
                    2 \mu_i + \sigma^2 + \gamma \sigma^2
                \right)
                \left(
                    1 - \gamma \sigma^4 / \mu_i^2
                \right)
            \end{aligned}
                &\text{when $\mu_i, \, \mu_j > \gamma^{1/2} \sigma^2$,} \\
            0
                &\text{otherwise.}
        \end{cases}
    \]
    When $\sigma_{ii} = 2 \mu_i^2$, 
    and $\mu_i > \sqrt{\gamma} \sigma^2$, the variance of the $i$th
    component simplifies to
    \(
        \tilde \sigma_{ii} 
        = 
        2 (\mu_i + \sigma^2)^2 \left( 1 - \gamma \sigma^4 / \mu_i^2 \right).
    \)
\end{theorem}

Next, we give the result for the singular vectors:
\begin{theorem}\label{T:vectors}
    Say Assumptions~\ref{A:factors}~--~\ref{A:noise} hold.  Then the
    $k\times k$ matrix $\mR_n \equiv \mV_n^\trans \mhV_n$ converges almost 
    surely to a matrix $\mR = \diag(r_1, r_2, \ldots, r_k)$, where
    \begin{equation}
        |r_i|^2
        =
        \begin{cases}
            \frac{ 1 - \gamma \sigma^4 / \mu_i^2 }
                 { 1 + \gamma \sigma^2 / \mu_i  }
            &\text{when $\mu_i > \sqrt{\gamma} \sigma^2$,} \\
            0
            &\text{otherwise.}
        \end{cases}
    \end{equation}
\end{theorem}


\subsection{Notes}

Some discussion of the assumptions and the results are in order:

\begin{enumerate}
    
\item
Assumptions~\ref{A:factors}~and~\ref{A:sizes} are simpler than the assumptions given in many other papers while still being quite general.  For example, Paul's ``spiked'' covariance model has data of the form
\[
    \mX = \mZ \mTheta^\trans + \mE,
\]
where $\mTheta$ is an $p \times k$ matrix of factors and $\mZ$ is an $n\times k$ matrix of factor loadings whose rows are \iid multivariate $\Normal(0,\, \mC)$ random variables for covariance matrx $\mC$ having eigen-decomposition $\mC = \mQ \mLambda \mQ^\trans$.  Letting $\mZ = \sqrt{n} \mhP \mhLambda \mhQ^\trans$ be the SVD of $\mZ$, Anderson's results \cite{anderson1963atp} give us that $\mhLambda$ and $\mhQ$ converge almost surely to $\mLambda$ and $\mQ$, respectively, and that the diagonal elements of $\sqrt{n} (\mhLambda - \mLambda)$ converge to a mean-zero multivariate normal whenever $\mC$ has no repeated eigenvalues.  If we define $\mU = \mhP$, $\mD = \mhLambda^\half$, and $\mV = \mTheta \mhQ$, then $\mX = \sqrt{n} \mU \mD \mV^\trans + \mE$, where the factors satisfy Assumptions~\ref{A:factors}~and~\ref{A:sizes}.  Dropping the normality assumption on the rows of $\mZ$ poses no problem.  Moreover, we can suppose instead of \iid that the rows of $\mZ$ be a martingale difference array with well-behaved low-order moments and still perform a transformation of the variables to get factors of the form we need for Theorems~\ref{T:values}~and~\ref{T:vectors}.

\item
There is a sign-indeterminancy in the sample and population singular vectors. Since we only give results for the squares $\mR_n^\trans \mR$ the signs do not matter and we choose them arbitrarily.

\item 
Most of the results in Theorems~\ref{T:values}~and~\ref{T:vectors} can be gotten from Onatski's results \cite{onatski}.   However, Onatski does not show that $\sqrt{n} (\hat \mu_i - \tilde \mu_i) \toP 0$ when $\mu_i$ is below the critical threshold.  Furthermore, Onatski proves convergence in probability, not almost sure convergence.

\end{enumerate}

\section{Preliminaries}\label{S:lowrank-preliminaries}

Without loss of generality we will assume that $\sigma^2 = 1$.  For now, we will also assume that $\gamma \geq 1$.

\subsection{Change of basis}

A convenient choice of basis will make it easier to study the SVD of $\mX_n$.  Define $\mU_{n,1} = \mU_n$, and choose $\mU_{n,2}$ so that
\(
    \left( 
    \begin{matrix}
        \mU_{n,1} & \mU_{n,2}
    \end{matrix}
    \right)
\) 
is an orthogonal matrix. Similarly, put $\mV_{n,1} = \mV_n$ and choose
$\mV_{n,2}$ so that
\(
    \left(
    \begin{matrix}
        \mV_{n,1} & \mV_{n,2}
    \end{matrix}
    \right)
\)
is orthogonal.  If we define
\(
    \mtE_{n,ij} = \mU_{n,i}^\trans \mE_n \mV_{n,j}
\)
and
\(
    \mX_{n,ij} = \mU_{n,i}^\trans \mX_n \mV_{n,j},
\)
then in block form,
\[
    \left( 
    \begin{matrix}
        \mU_{n,1}^\trans \\
        \mU_{n,2}^\trans
    \end{matrix}
    \right)
    \mX_n
    \left(
    \begin{matrix}
        \mV_{n,1} & \mV_{n,2}
    \end{matrix}
    \right)
    =
    \left(
    \begin{matrix}
        \sqrt{n} \mD_n + \mtE_{n,11} & \mtE_{n,12} \\
        \mtE_{n,21}                  & \mtE_{n,22}
    \end{matrix}
    \right).
\]
Because Gaussian white noise is orthogonally invariant, the
matrices $\mtE_{n,ij}$ are all independent with \iid $\Normal(0, \, 1)$ elements.  Let
\[
    \mtE_{n,22}
    =
    \sqrt{n}
    \left(
    \begin{matrix}
        \mO_{n,1} & \mO_{n,2}
    \end{matrix}
    \right)
    \left(
    \begin{matrix}
        \mLambda_n^{1/2} \\
        0
    \end{matrix}
    \right)
    \mP_{n}^\trans
\]
be the SVD of $\mtE_{n,22}$, with
\(
    \mLambda_n
    =
    \diag \left(
        \lambda_{n,1}, 
        \lambda_{n,2}, 
        \ldots
        \lambda_{n,p-k}
    \right)
\)
.  Note that 
\(
    \mtE_{n,22}^\trans \mtE_{n,22}
    \sim
    \Wishart_{p-k} \left( n-k, \mI_{p-k} \right).
\)
Define
\begin{align*}
    \mtX_n
        &=
            \left(
            \begin{matrix}
                \mI_k & 0 \\
                0  & \mO_{n,1}^\trans \\
                0  & \mO_{n,2}^\trans
            \end{matrix}
            \right)
            \left(
            \begin{matrix}
                \sqrt{n} \mD_n + \mtE_{n,11} & \mtE_{n,12} \\
                \mtE_{n,21}                  & \mtE_{n,22}
            \end{matrix}
            \right)
            \left(
            \begin{matrix}
                \mI_k & 0 \\
                0     & \mP_n
            \end{matrix}
            \right) \\
        &=
            \left(
            \begin{matrix}
                \sqrt{n} \mD_n + \mE_{n,11} & \mE_{n,12} \\
                \mE_{n,21}                  & \sqrt{n} \mLambda^{1/2}_n \\
                \mE_{n,31}                  & 0
            \end{matrix}
            \right),
\end{align*}
where 
$\mE_{n,11} = \mtE_{n,11}$, 
$\mE_{n,12} = \mtE_{n,12} \mP_n$,
$\mE_{n,21} = \mO_{n,1}^\trans \mtE_{n,21}$, and
$\mE_{n,31} = \mO_{n,2}^\trans \mtE_{n,31}$.
Finally, let
\[
    \mtX_n
    \approx
    \mtU_n \mtD_n \mtV_n
\]
be the SVD of $\mtX_n$, truncated to $k$ terms.  Lastly, put the left and right singular vectors in block form as
\[
    \mtU_n
    =
    \left(
    \begin{matrix}
        \mtU_{n,1} \\
        \mtU_{n,2}
    \end{matrix}
    \right)
\]
and
\[
    \mtV_n
    =
    \left(
    \begin{matrix}
        \mtV_{n,1} \\
        \mtV_{n,2}
    \end{matrix}
    \right),
\]
where $\mtU_{n,1}$ and $\mtV_{n,1}$ both $k\times k$ matrices.

We have gotten to $\mtX_n$ via an orthogonal change of basis applied to $\mX_n$.  By carefully choosing this basis, we have assured that:
\begin{enumerate}
    \item The blocks of $\mtX_n$ are all independent.
    \item The elements of the matrices $\mE_{n,ij}$ are 
        \iid $\Normal\left( 0, 1 \right)$.
    \item The elements 
        $\{ n \lambda_{n,1}, n \lambda_{n,2}, \ldots, 
            n \lambda_{n,p-k} \}$ are eigenvalues from
        a white Wishart matrix with $n-k$ degrees of freedom.
    \item $\mtX_n$ and $\mX_n$ have the same singular values.  This implies
        that $\mhD_n = \mtD_n$.
    \item The left singular vectors of $\mX_n$ can be recovered from the
        left singular vectors of $\mtX_n$ via multiplication by an orthogonal
        matrix.  The same holds for the right singular vectors.
    \item The dot product matrix $\mU_n^\trans \mhU_n$ is equal to
        $\mtU_{n,1}$.
        Similarly, $\mV_n^\trans \mhV_n = \mtV_{n,1}$.
\end{enumerate}

\subsection{The secular equation}

We set $\mS_n = \frac{1}{n} = \mtX_n^\trans \mtX_n$.  The eigenvalues and eigenvectors of $\mS_n$ are the squares of the singular values of $\mtX_n$ and its right singular vectors, respectively.  In block form, we have
\[
    \mS_n
    =
    \left(
    \begin{matrix}
        \mS_{n,11} & \mS_{n,12} \\
        \mS_{n,21} & \mS_{n,22}
    \end{matrix}
    \right),
\]
where
\begin{align*}
    \begin{split}
    \mS_{n,11}
        &=
            \mD_n^2 
            + 
            \frac{1}{\sqrt{n}} 
            \left( 
                \mD_n \mE_{n,11} + \mE_{n,11}^\trans \mD_n
            \right) \\
            &\phantom{= \mD_n^2} +
            \frac{1}{n}
            \left(
                \mE_{n,11}^\trans \mE_{n,11}
                +
                \mE_{n,21}^\trans \mE_{n,21}
                +
                \mE_{n,31}^\trans \mE_{n,31}
            \right),
    \end{split} \\
    \mS_{n,12}
        &=
            \frac{1}{\sqrt{n}}
            \left(
                \mD_n \mE_{n,12} + \mE_{n,21}^\trans \mLambda_n^{1/2}
            \right)
            +
            \frac{1}{n}
            \mE_{n,11}^\trans \mE_{n,12}, \\
    \mS_{n,21}
        &=
            \frac{1}{\sqrt{n}}
            \left(
                \mE_{n,12}^\trans \mD_n
                +
                \mLambda_n^{1/2} \mE_{n,21}
            \right)
            +
            \frac{1}{n}
            \mE_{n,12}^\trans \mE_{n,11}, \\
    \mS_{n,22}
        &=
            \mLambda_n
            +
            \frac{1}{n}
            \mE_{n,12}^\trans \mE_{n,12}.
\end{align*}

Now we study the eigendecomposition of $\mS_n$.  If 
\(
    \vx
    = 
    \left( 
    \begin{matrix}
        \vx_{1} \\
        \vx_{2}
    \end{matrix}
    \right)
\)
is an eigenvector of $\mS_n$ with eigenvalue $\mu$, then
\[
    \left(
    \begin{matrix}
        \mS_{n,11} & \mS_{n,12} \\
        \mS_{n,21} & \mS_{n,22}
    \end{matrix}
    \right)
    \left( 
    \begin{matrix}
        \vx_{1} \\
        \vx_{2}
    \end{matrix}
    \right)
    =
    \mu
    \left( 
    \begin{matrix}
        \vx_{1} \\
        \vx_{2}
    \end{matrix}
    \right).
\]
If $\mu$ is not an eigenvalue of $\mS_{n,22}$, then we get
\begin{subequations}
\begin{equation}\label{E:x2-from-x1}
    \vx_2
    =
    -
    \left(
        \mS_{n,22}
        -
        \mu
        \mI_{p-k}
    \right)^{-1}
    \mS_{n,21} \,
    \vx_1
\end{equation}
and
\begin{equation}\label{E:secular-nof}
    \left(
        \mS_{n,11}
        -
        \mu
        \mI_k
        -
        \mS_{n,12}
        \left(
            \mS_{n,22}
            -
            \mu
            \mI_{p-k}
        \right)^{-1}
        \mS_{n,21}
    \right)
    \vx_1
    =
    0.
\end{equation}
\end{subequations}
Conversely, if $(\mu, \vx_1)$ is a pair that solves~\eqref{E:secular-nof}
and $\vx_1 \neq 0$, then
\begin{equation}\label{E:x-from-x1}
    \vx
    =
    \left(
    \begin{matrix}
        \vx_1 \\
        -
        \left(
            \mS_{n,22}
            -
            \mu
            \mI_{p-k}
        \right)^{-1}
        \mS_{n,21} \,
        \vx_1
    \end{matrix}
    \right)
\end{equation}
is an eigenvector of $\mS_n$ with eigenvalue $\mu$.  We define
\begin{gather}
    \mT_n(z)
        =
            \left(
                \mS_{n,11}
                -
                z
                \mI_k
                -
                \mS_{n,12}
                \left(
                    \mS_{n,22}
                    -
                    z
                    \mI_{p-k}
                \right)^{-1}
                \mS_{n,21}
            \right), \label{E:secular-T} \\
    f_n ( z, \vx )
    =
    \mT_n(z) \,
    \vx, \label{E:secular-f}
\end{gather}
and refer to the equation $f_n(z, \vx) = 0$ as the \emph{secular equation}.

\subsection{Outline of the proof}

We first show that with probability one, $\mS_n$ and $\mS_{n,22}$ have no eigenvalues in common.  This implies that all of the eigenvalues of $\mS_n$ are solutions to the secular equation.  Next, we study the solutions of 
$f_n(z, \vx) = 0$.  It turns out that these solutions are dramatically different depending on the size of $z$.  If $z > (1 + \gamma^{-1/2})^2$,
then we can write
\begin{align*}
    z 
        &= 
            z^0 
            + 
            \frac{1}{\sqrt{n}}
            z^1 
            + 
            \ohP\left( \frac{1}{\sqrt{n}} \right), \\
    \vx
        &=
            \vx^0
            +
            \frac{1}{\sqrt{n}}
            \vx^1
            + 
            \ohP\left( \frac{1}{\sqrt{n}} \right).
\end{align*}
Otherwise, the solution is
\begin{align*}
    z
        &=
            z^0
            + 
            \ohP\left( \frac{1}{\sqrt{n}} \right), \\
    \vx
        &=
            \vx^0
            +
            \ohP\left( \frac{1}{\sqrt{n}} \right).
\end{align*}
Once we have a solution to the secular equation, we can construct an
eigenvector of $\mS_n$ using equation~\eqref{E:x-from-x1}.  Finally, we will derive the left singular vectors from the right by multiplying by $\mtX_n$.

\section{Analysis of the secular equation}

By making use of the Sherman-Morrison-Wordbury formula~\cite{golub1996mc}, we can simplify $\mT_n(z)$.  We first write
\begin{align*}
    \left(
        \mS_{n,22} - z \mI_{p-k}
    \right)^{-1}
        &=
            \left(
                \left(
                    \mLambda_n - z \mI_{p-k}
                \right)
                +
                \frac{1}{n}
                \mE_{n,12}^\trans \mE_{n12}
            \right)^{-1} \\
        \begin{split}
        &=
            \left(
                \mLambda_n
                -
                z
                \mI_{p-k}
            \right)^{-1} \\
            &\qquad-
            \frac{1}{n}
            \left(
                \mLambda_n
                -
                z
                \mI_{p-k}
            \right)^{-1} \\
            &\qquad\qquad\cdot
            \mE_{n,12}^\trans
            \left(
                \mI_k
                +
                \frac{1}{n}
                \mE_{n,12}
                \left(
                    \mLambda_n
                    -
                    z
                    \mI_{p-k}
                \right)^{-1}
                \mE_{n,12}^\trans
            \right)^{-1}
            \mE_{n,12} \\
            &\qquad\qquad\cdot
            \left(
                \mLambda_n
                -
                z
                \mI_{p-k}
            \right)^{-1}.
    \end{split}
\end{align*}
If we define $\mtD_n = \mD_n + \frac{1}{\sqrt{n}} \mE_{n,11}$, then we can compute
\begin{align*}
    \begin{split}
    \mS_{n,12} 
    \Big(
        \mS_{n,22} &- z \mI_{p-k}
    \Big)^{-1}
    \mS_{n,21} \\
        &=
            \frac{1}{n}
            \left(
                \mtD_n^\trans \mE_{n,12}
                +
                \mE_{n,21}^\trans \mLambda_n^{1/2}
            \right) \\
            &\qquad\cdot
            \left(
                \mS_{n,22} - z \mI_{p-k}
            \right)^{-1} \\
            &\qquad\cdot
            \left(
                \mE_{n,12}^\trans \mtD_n
                +
                \mLambda_n^{1/2} \mE_{n,21}
            \right)
    \end{split} \\
    \begin{split}
        &=
            \frac{1}{n}
            \mtD_n^\trans
            \mE_{n,12}
            \left(
                \mLambda_n
                -
                z
                \mI_{p-k}
            \right)^{-1}
            \mE_{n,12}^\trans
            \mtD_n \\
            &\qquad+
            \frac{1}{n}
            \mtD_n^\trans
            \mE_{n,12}
            \left(
                \mLambda_n
                -
                z
                \mI_{p-k}
            \right)^{-1}
            \mLambda_n^{1/2}
            \mE_{n,21} \\
            &\qquad+
            \frac{1}{n}
            \mE_{n,21}^\trans
            \mLambda_n^{1/2}
            \left(
                \mLambda_n
                -
                z
                \mI_{p-k}
            \right)^{-1}
            \mE_{n,12}^\trans
            \mtD_n \\
            &\qquad+
            \frac{1}{n}
            \mE_{n,21}^\trans
            \mLambda_n^{1/2}
            \left(
                \mLambda_n
                -
                z
                \mI_{p-k}
            \right)^{-1}
            \mLambda_n^{1/2}
            \mE_{n,21} \\
            &\qquad-
            \frac{1}{n^2}
            \mtD_n^\trans
            \mE_{n,12}
            \left(
                \mLambda_n
                -
                z
                \mI_{p-k}
            \right)^{-1}
            \mE_{n,12}^\trans \\
            &\qquad\qquad\cdot
            \left(
                \mI_k
                +
                \frac{1}{n}
                \mE_{n,12}
                \left(
                    \mLambda_n
                    -
                    z
                    \mI_{p-k}
                \right)^{-1}
                \mE_{n,12}^\trans
            \right)^{-1} \\
            &\qquad\qquad\cdot
            \mE_{n,12}
            \left(
                \mLambda_n
                -
                z
                \mI_{p-k}
            \right)^{-1}
            \mE_{n,12}^\trans
            \mtD_n \\
            &\qquad-
            \frac{1}{n^2}
            \mtD_n^\trans
            \mE_{n,12}
            \left(
                \mLambda_n
                -
                z
                \mI_{p-k}
            \right)^{-1}
            \mE_{n,12}^\trans \\
            &\qquad\qquad\cdot
            \left(
                \mI_k
                +
                \frac{1}{n}
                \mE_{n,12}
                \left(
                    \mLambda_n
                    -
                    z
                    \mI_{p-k}
                \right)^{-1}
                \mE_{n,12}^\trans
            \right)^{-1} \\
            &\qquad\qquad\cdot
            \mE_{n,12}
            \left(
                \mLambda_n
                -
                z
                \mI_{p-k}
            \right)^{-1}
            \mLambda_n^{1/2}
            \mE_{n,21} \\
            &\qquad-
            \frac{1}{n^2}
            \mE_{n,21}^\trans
            \mLambda_{n}^{1/2}
            \left(
                \mLambda_n
                -
                z
                \mI_{p-k}
            \right)^{-1}
            \mE_{n,12}^\trans \\
            &\qquad\qquad\cdot
            \left(
                \mI_k
                +
                \frac{1}{n}
                \mE_{n,12}
                \left(
                    \mLambda_n
                    -
                    z
                    \mI_{p-k}
                \right)^{-1}
                \mE_{n,12}^\trans
            \right)^{-1} \\
            &\qquad\qquad\cdot
            \mE_{n,12}
            \left(
                \mLambda_n
                -
                z
                \mI_{p-k}
            \right)^{-1}
            \mE_{n,12}^\trans
            \mtD_n \\
            &\qquad-
            \frac{1}{n^2}
            \mE_{n,21}^\trans
            \mLambda_{n}^{1/2}
            \left(
                \mLambda_n
                -
                z
                \mI_{p-k}
            \right)^{-1}
            \mE_{n,12}^\trans \\
            &\qquad\qquad\cdot
            \left(
                \mI_k
                +
                \frac{1}{n}
                \mE_{n,12}
                \left(
                    \mLambda_n
                    -
                    z
                    \mI_{p-k}
                \right)^{-1}
                \mE_{n,12}^\trans
            \right)^{-1} \\
            &\qquad\qquad\cdot
            \mE_{n,12}
            \left(
                \mLambda_n
                -
                z
                \mI_{p-k}
            \right)^{-1}
            \mLambda_{n}^{1/2}
            \mE_{n,21}
    \end{split}
\end{align*}
To put it mildly, this is a formidable expression.  We can simplify the formula substantially when $z > \left(1 + \gamma^{-1/2} \right)^2$.

\begin{lemma}\label{L:eij-product-limits}
    If $z > \left(1 + \gamma^{-1/2} \right)^2$, then
    \begin{align*}
        \frac{1}{n}
        \mE_{n,12}
        \left(
            \mLambda_n
            -
            z \mI_{p-k}
        \right)^{-1}
        \mE_{n,12}^\trans
            &\toas
                \gamma^{-1}
                \int
                    \frac{1}{t - z} d\FMP_\gamma(t)
                \cdot
                \mI_k, \\
        \frac{1}{n}
        \mE_{n,12}
        \left(
            \mLambda_n
            -
            z \mI_{p-k}
        \right)^{-1}
        \mLambda_n^{1/2}
        \mE_{n,21}
            &\toas
                0, \\
        \frac{1}{n}
        \mE_{n,21}^\trans
        \left(
            \mLambda_n
            -
            z \mI_{p-k}
        \right)^{-1}
        \mE_{n,21}
            &\toas
                \gamma^{-1}
                \int
                    \frac{1}{t - z} d\FMP_\gamma(t)
                \cdot
                \mI_k, \\
        \frac{1}{n}
        \mE_{n,12} \mE_{n,12}^\trans
            &\toas
            \gamma^{-1}
            \mI_k, \\
        \frac{1}{n}
        \mE_{n,21}^\trans \mE_{n,21}
            &\toas
                \gamma^{-1}
                \mI_k, \\
        \frac{1}{n}
        \mE_{n,31}^\trans \mE_{n,31}
            &\toas
                \left( 1 - \gamma^{-1} \right)
                \mI_k.
    \end{align*}
\end{lemma}
\begin{proof}
We prove the result for the first quantity and the other derivations are analogous.  We have that
\begin{align*}
    \left(
        \frac{1}{n}
        \mE_{n,12}
        \left(
            \mLambda_n
            -
            z \mI_{p-k}
        \right)^{-1}
        \mE_{n,12}^\trans    
    \right)_{ij}
    =
    \frac{p-k}{n}
    \cdot
    \frac{1}{p-k}
    \sum_{\alpha=1}^{p-k}
        \frac{ \left( \mE_{n,12} \right)_{i\alpha}
               \left( \mE_{n,12} \right)_{j\alpha}
             }
             { \lambda_{n,\alpha} - z }
\end{align*}
Let $N = p-k$, define weights 
\(
    W_{N,\alpha} = (\lambda_{n,\alpha} - z)^{-1},
\)
and let
\(
    Y_{N,\alpha} = \left( \mE_{n,12} \right)_{i\alpha}
                   \left( \mE_{n,12} \right)_{j\alpha}.
\)
Letting $b(\gamma) = \left( 1 + \gamma^{-1/2} \right)^2$, the function
\[
    g(t) 
    = 
    \begin{cases}
        \frac{1}{t - z} 
            &\text{if $t \leq b(\gamma)$,} \\
        \frac{1}{b(\gamma) - z}
            &\text{otherwise,}
    \end{cases}
\]
is bounded and continuous.  Moreover, since $\lambda_{n,1} \toas b(\gamma)$, with probability one $g(\lambda_{n,\alpha})$ is eventually equal to $W_{N,\alpha}$ for all $\alpha$.  The Wishart LLN (Corollary~\ref{C:wishart-lln}) gives us that
\[
    \frac{1}{N}
    \sum_{i=\alpha}^N W_{N,\alpha}
        \toas
            \int
                \frac{1}{t - z} d\FMP_\gamma(t).
\]
Since $| W_{N,\alpha} | \leq W_{N,1} \overset{\text{a.s.}}{\leq} b(\gamma)$, the fourth moments of the weights are uniformly bounded in $N$.  

The $Y_{N,\alpha}$ are all \iid with $\E Y_{N,\alpha} = \delta_{ij}$ and $\E Y_{N,\alpha}^4 < \infty$.  The weighed SLLN (Proposition~\ref{P:weighted-slln}) gives us that
\[
    \frac{1}{N}
    \sum_{\alpha=1}^N
        W_{N,\alpha}
        Y_{N,\alpha}
    \toas
        \delta_{ij}
        \int
            \frac{1}{t - z} d\FMP_\gamma(t).
\]
Since $\frac{p-k}{n} \to \gamma^{-1}$, this completes the proof.
\end{proof}

As an aside, we remark that the quantity
\begin{subequations}
\begin{align}
    m(z)
        &\define
            \int
                \frac{1}{t - z} d\FMP_\gamma(t) \notag \\
        &=
            \gamma
            \cdot
            \frac{ -(z - 1 + \gamma^{-1})
                   +\sqrt{ \big( z - b(\gamma) \big) 
                           \big( z - a(\gamma) \big) } }
                 { 2 z },
\end{align}
is equal to the Stieltjes transform of $\FMP_\gamma$, where
\begin{align}
    a(\gamma) &= \left( 1 - \gamma^{-1/2} \right)^2, \\
    b(\gamma) &= \left( 1 + \gamma^{-1/2} \right)^2.
\end{align}
\end{subequations}
It has inverse
\begin{equation}
    z(m)
        = 
            -
            \frac{1}{m}
            +
            \frac{1}{1 + \gamma^{-1} \, m}.
\end{equation}

\begin{lemma}\label{L:eij-product-limits-uniform}
    Considered as functions of $z$, the quantities in 
    Lemma~\ref{L:eij-product-limits} converge uniformly over any closed
    interval $[u,v] \subset \big( b(\gamma),\infty \big)$.
\end{lemma}
\begin{proof}
    We show this for the first quantity and the other proofs are similar.
    Defining
    \[
        e_{n,ij}(z)
        =
        \left(
            \frac{1}{n}
            \mE_{n,12}
            \left(
                \mLambda_n - z \mI_{p-k}
            \right)^{-1}
            \mE_{n,12}^\trans
        \right)_{ij},
    \]
    we will show that for all $ij$ with $1 \leq i,j \leq k$, 
    \[
        \sup_{z \in [u,v]} 
            \left|
                e_{n,ij}(z)
                    -
                    \gamma^{-1}
                    m(z)
                    \delta_{ij}
            \right|
            \toas 0.
    \]
    Let $\epsilon > 0$ be arbitrary.  Note that for 
    $z > b(\gamma)$, $m'(z) > 0$ and $m''(z) < 0$.  We let $d = m'(u)$ and
    choose a grid
    \[
        u = z_1 < z_2 < \cdots < z_{M-1} < z_M = v,
    \]
    with $|z_l - z_{l+1}| = \frac{\gamma \epsilon}{2 d}$.  
    Then $|\gamma^{-1} m(z_l) - \gamma^{-1} m(z_{l+1})| < \frac{\epsilon}{2}$.  
    From 
    Lemma~\ref{L:eij-product-limits}, we can find $N$ large enough such
    that for $n > N$, 
    \[
        \max_{l\in\{1, \ldots, M\}}
            \left|
                e_{n,ij}(z_l)
                    -
                    \gamma^{-1}
                    m(z_l)
                    \delta_{ij}
            \right|
            <
            \frac{\epsilon}{2}.
    \]
    Also guarantee that $N$ is large enough so that $\lambda_{n,1} < u$ 
    (this is possible since $\lambda_{n,1} \toas b(\gamma) < u$).
    Let $z \in [u,v]$ be arbitrary and find $l$ such that 
    $z_l \leq z \leq z_{l+1}$.
    Observe that $e_{n,ij}(z)$ is monotone for $z > \lambda_{n,1}$.  Thus,
    either
    \[
        e_{n,ij}(z_l) \leq e_{n,ij}(z) \leq e_{n,ij}(z_{l+1})
    \]
    or
    \[
        e_{n,ij}(z_{l+1}) \leq e_{n,ij}(z) \leq e_{n,ij}(z_{l}).
    \]
    If $i \neq j$, we have for $n > N$,
    \[
        -\frac{\epsilon}{2} < e_{n,ij}(z) < \frac{\epsilon}{2},
    \]
    For $i = j$, $e_{n,ij}(z)$ is 
    monotonically increasing and
    \[
        \gamma^{-1} m(z_l) - \frac{\epsilon}{2}
            <
                e_{n,ij}(z)
                    <
                        \gamma^{-1} m(z_{l+1}) + \frac{\epsilon}{2},
    \]
    so
    \[
        \gamma^{-1} m(z) - \epsilon
            <
                e_{n,ij}(z)
                    <
                        \gamma^{-1} m(z) + \epsilon.
    \]
    In either case, 
    $|e_{n,ij}(z) - \gamma^{-1} m(z) \delta_{ij}| < \epsilon$.
\end{proof}

In fact we can get the second-order behavior of the quantities of Lemma~\ref{L:eij-product-limits}.

\begin{lemma}\label{L:eij-product-scaled-limits}
    If $z_1, z_2, \ldots, z_l > \left( 1 + \gamma^{-1/2} \right)^2$, 
    then jointly for $z \in \{ z_1, z_2, \ldots, z_l \}$
    \begin{align*}
        \sqrt{n}
        \left(
            \frac{1}{n}
            \mE_{n,12}
            \left(
                \mLambda_n
                -
                z
                \mI_{p-k}
            \right)^{-1}
            \mE_{n,12}^\trans
            -
            \gamma^{-1}
            m(z)
            \mI_k
        \right)
            &\tod \mF(z), \\
        \sqrt{n}
        \left(
            \frac{1}{n}
            \mE_{n,12}
            \left(
                \mLambda_n
                -
                z
                \mI_{p-k}
            \right)^{-1}
            \mLambda_{n}^{1/2}
            \mE_{n,21}
        \right)
            &\tod \mG(z), \\
        \sqrt{n}
        \left(
            \frac{1}{n}
            \mE_{n,21}^\trans
            \left(
                \mLambda_n
                -
                z
                \mI_{p-k}
            \right)^{-1}
            \mE_{n,21}
            -
            \gamma^{-1}
            m(z)
            \mI_k
        \right)
            &\tod \mH(z), \\
        \sqrt{n}
        \left(
            \frac{1}{n}
            \mE_{n,12} \mE_{n,12}^\trans
            -
            \gamma^{-1}
            \mI_k
        \right)
            &\tod \mK, \\
        \sqrt{n}
        \left(
            \frac{1}{n}
            \mE_{n,21}^\trans \mE_{n,21}
            -
            \gamma^{-1}
            \mI_k
        \right)
            &\tod \mL_1, \\
        \sqrt{n}
        \left(
            \frac{1}{n}
            \mE_{n,31}^\trans \mE_{n,31}
            -
            \left( 1 - \gamma^{-1} \right)
            \mI_k
        \right)
            &\tod \mL_2,
    \end{align*}
    where $\mF(z)$, $\mG(z)$, $\mH(z)$, $\mK$, $\mL_1$ and $\mL_2$ are 
    defined on the same probability space.  
    All but $\mG(z)$ are symmetric. Jointly, their elements are
    distributed as a multivariate normal.  If we define
    \[
        \mathcal{M}
        =
        \Big\{ \mK, \mL_1, \mL_2 \Big\}
        \cup
        \Bigg(
        \bigcup_{i=1}^l
            \Big\{ 
                \mF(z_i), \mG(z_i), \mH(z_i)
            \Big\}
        \Bigg),
    \] 
    then for $\mA, \mB \in \mathcal{M}$, $(i,j) \neq (i',j')$, and 
    $(i,j) \neq (j',i')$,
    \[
        \cov\left( \mA_{ij}, \mB_{i'j'} \right)
        =
        0.
    \]
    The sets
    \[
        \Big\{ \mK, \mF( z ) \Big\},
        \Big\{ \mG( z) \Big\},
        \Big\{ \mL_1, \mH( z )\Big\},
        \Big\{ \mL_2 \Big\}
    \]
    are mutually independent of each other.
    The other covariances are defined by
    \begin{align*}
        \cov\Big( F_{ij}(z_1), \, F_{ij}(z_2) \Big)
            &= \gamma^{-1}
               \left( 1 + \delta_{ij} \right)
               \frac{ m(z_1) - m(z_2) }
                    { z_1 - z_2 }, \\
        \cov\Big( F_{ij}(z_1), \, K_{ij} \Big)
            &= \gamma^{-1}
               \left( 1 + \delta_{ij} \right) m(z_1), \\
        \cov\Big( K_{ij}, \, K_{ij} \Big)
            &= \gamma^{-1}
               \left( 1 + \delta_{ij} \right), \\
        \cov\Big( G_{ij}(z_1), \, G_{ij}(z_2) \Big)
            &= \gamma^{-1} \,
               \frac{ z_1 \, m(z_1) - z_2 \, m(z_2) }
                    { z_1 - z_2 }, \\
        \cov\Big( H_{ij}(z_1), \, H_{ij}(z_2) \Big)
            &= \gamma^{-1}
               \left( 1 + \delta_{ij} \right)
               \frac{ m(z_1) - m(z_2) }
                    { z_1 - z_2 }, \\
        \cov\Big( H_{ij}(z_1), \, L_{1,ij} \Big)
            &= \gamma^{-1}
               \left( 1 + \delta_{ij} \right) m(z_1), \\
        \cov\Big( L_{1,ij}, \, L_{1,ij} \Big)
            &= \gamma^{-1}
               \left( 1 + \delta_{ij} \right), \\
        \cov\Big( L_{2,ij}, \, L_{2,ij} \Big)
            &= (1 - \gamma^{-1})
               \left( 1 + \delta_{ij} \right),
    \end{align*}
    with the interpretation
    \begin{align*}
        \frac{ m(z_1) - m(z_2) }
             { z_1 - z_2 }
            &=
                m'(z_1), \\
    \intertext{and}
        \frac{ z_1 \, m(z_1) - z_2 \, m(z_2) }
             { z_1 - z_2 }
            &= m( z_1 ) + z_1 \, m'(z_1)
    \end{align*}
    for $z_1 = z_2$.
\end{lemma}
\begin{proof}
    The distributions of $\mK$, $\mL_1$, and $\mL_2$ follow from the 
    multivariate CLT.  For example, we have
    \begin{align*}
        \sqrt{n}
        \left(
            \frac{1}{n} \mE_{n,12} \mE_{n,12}^\trans
            -
            \gamma^{-1}
            \mI_k
        \right)
            &=
                \left(
                    \gamma^{-1}
                    \sqrt{ \frac{ n }{ p - k } }
                \right)
                \cdot
                \left(
                    \sqrt{ p - k }
                    \left[
                        \frac{1}{p-k}
                        \mE_{n,12} \mE_{n,12}^\trans
                        -
                        \mI_k
                    \right]
                \right) \\
            &\tod
                \mK.
    \end{align*}
    For the other limits, we will need the Wishart CLT 
    (Theorem~\ref{T:wishart-clt}) and the strong weighted multivariate CLT
    (Corollary~\ref{C:strong-weighted-clt}).  To save space we only
    give the argument for the joint distribution of $\mF(z_1)$ and 
    $\mF(z_2)$.
    
    Put $N = p-k$ and consider the $2 k^2$-dimensional vector
    \(
        \vY_{N,\alpha}
        =
        \left(
        \begin{matrix}
            \vtY_{N,\alpha} \\
            \vtY_{N,\alpha}
        \end{matrix}
        \right),
    \)
    where
    \(
        \vtY_{N,\alpha}
        =
        \vecm \left(
            \vE_{n,12,\alpha} \, \vE_{n,12,\alpha}^\trans
        \right)
    \)
    and
    \(
        \mE_{n,12}
        =
        \left(
        \begin{matrix}
            \vE_{n,12,1} &
            \vE_{n,12,2} &
            \cdots &
            \vE_{n,12,N} &            
        \end{matrix}
        \right).
    \)
    Define the $2 k^2$-dimensional weight vector
    \(
        \vW_{N,\alpha}
        =
        \left(
        \begin{matrix}
            \vW_{N,\alpha,1} \\
            \vW_{N,\alpha,2}
        \end{matrix}
        \right),
    \)
    where
    \(
        \vW_{N,\alpha,i}
        =
        \frac{1}{\lambda_\alpha - z_i}
        \vone.
    \)
    We have that for $\alpha = 1, 2, \ldots, N$, the $\vY_{N,\alpha}$ are
    \iid with
    \[
        \E \left[ \vY_{N,1} \right]
        =
        \vmu^Y
        =
        \left(
        \begin{matrix}
            \vtmu^Y \\
            \vtmu^Y
        \end{matrix}
        \right)
    \]
    and $\vtmu^Y = \vecm\left( \mI_k \right)$.  Also, we have
    \[
        \E \left[ 
            \left(
                \vY_{N,1}
                -
                \vmu^Y
            \right)
            \left(
                \vY_{N,1}
                -
                \vmu^Y
            \right)^\trans
        \right]
        =
        \mSigma^Y
        =
        \left(
        \begin{matrix}
            \mtSigma^Y & \mtSigma^Y \\
            \mtSigma^Y & \mtSigma^Y
        \end{matrix}
        \right),
    \]
    where
    \(
        \mtSigma^Y
        =
        \E \Big[
            \Big( \!
                \vecm\left( 
                    \vE_{n,12,1} \vE_{n,12,1}^\trans - \mI_k
                \right) \! \!
            \Big)
            \Big( \!
                \vecm\left( 
                    \vE_{n,12,1} \vE_{n,12,1}^\trans - \mI_k
                \right) \! \!
            \Big)^{\! \trans}
        \Big]
    \)
    is a $k^2 \times k^2$ matrix defined by the relation
    \[
        \E \left[
            \left( 
                \vE_{n,12,1} \vE_{n,12,1}^\trans - \mI_k
            \right)_{ij}
            \left( 
                \vE_{n,12,1} \vE_{n,12,1}^\trans - \mI_k
            \right)_{i'j'}
        \right]
        =
        \delta_{(i,j) = (i',j')}
        +
        \delta_{(i,j) = (j',i')}.
    \]
    That is, the diagonal elements of 
    $\vE_{n,12,1} \vE_{n,12,1}^\trans - \mI_k$ have variance $2$ and
    the off-diagonal elements have variance $1$.  Aside from the matrix
    being symmetric, the unique elements are all uncorrelated.
    
    Letting
    \begin{align*}
        \mu^W_i
            &=
                \int \frac{ d\FMP_\gamma(t) }
                          { t - z_i } \\
            &=
                m( z_i ), \\
        \sigma^W_{ij}
            &=
                \int \frac{ d\FMP_\gamma(t) }
                          { (t - z_i)(t - z_j) } \\
            &=
                \frac{ m( z_1 ) - m( z_2 ) }
                     { z_1 - z_2 },
    \end{align*}
    the Wishart LLN combined with the truncation argument of
    Lemma~\ref{L:eij-product-limits} gives us that
    \begin{align*}
        \frac{1}{N}
        \sum_{\alpha=1}^N
            \frac{1}{  \lambda_{n,\alpha} - z_i  }
            &\toas
                \mu^W_{i}, \\
        \frac{1}{N}
        \sum_{\alpha=1}^N
            \frac{1}{ ( \lambda_{n,\alpha} - z_i )
                      ( \lambda_{n,\alpha} - z_j ) }
            &\toas
                \sigma^W_{ij}.
    \end{align*}
    Thus,
    \begin{align*}
        \frac{1}{N}
        \sum_{\alpha=1}^N
            \vW_{N,\alpha}
            &\toas
                \vmu^W \\
            &=
                \left(
                \begin{matrix}
                    \mu^W_1 \vone \\
                    \mu^W_2 \vone
                \end{matrix}
                \right), \\
    \intertext{and}
        \frac{1}{N}
        \sum_{\alpha=1}^N
            \vW_{N,\alpha} \,
            \vW_{N,\alpha}^\trans
            &\toas
                \mSigma^W \\
            &=
                \left(
                \begin{matrix}
                    \sigma^W_{11} \vone \, \vone^\trans &
                        \sigma^W_{12} \vone \, \vone^\trans \\
                    \sigma^W_{21} \vone \, \vone^\trans &
                        \sigma^W_{11} \vone \, \vone^\trans
                \end{matrix}
                \right).
    \end{align*}
    Moreover, the Wishart CLT tells us that the error in each of the
    sums is of size $\OhP\left( \frac{1}{N} \right)$.  
        
    As in Lemma~\ref{L:eij-product-limits}, the fourth moments of
    $\vW_{N,\alpha}$ and $\vY_{N,\alpha}$ are all well-behaved.  Finally,
    we can invoke the strong weighted CLT 
    (Corollary~\ref{C:strong-weighted-clt}) to get that
    \[
        \sqrt{N} \left(
            \frac{1}{N}
            \sum_{\alpha=1}^N
                \vW_{N,\alpha} \bullet \vY_{N,\alpha}
            -
                \vmu^W \bullet \vmu^T
        \right)
    \]
    converges in distribution to a mean-zero multivariate normal with
    covariance
    \[
        \mSigma^W \bullet \mSigma^Y
            =
                \left(
                \begin{matrix}
                    \sigma^W_{11} \mtSigma &
                        \sigma^W_{12} \mtSigma \\
                    \sigma^W_{21} \mtSigma &
                        \sigma^W_{22} \mtSigma 
                \end{matrix}
                \right).
    \]
    This completes the proof since
    \begin{multline*}
        \sqrt{n}
        \vecm \left(
            \frac{1}{n}
            \mE_{n,12}
            \left(
                \mLambda_n
                -
                z_i
                \mI_{p-k}
            \right)^{-1}
            \mE_{n,12}^\trans
            -
            \gamma^{-1}
            m( \mu )
            \mI_k
        \right) \\
            =
                \gamma^{-1}
                \sqrt{ \frac{n}{p-k} }
                \cdot
                \sqrt{ N }
                \Bigg( 
                    \frac{1}{N}
                    \sum_{\alpha=1}^N
                        \vW_{N,\alpha,i} \bullet \vtY_{N,\alpha}
                    -
                        \vmu_i^W \bullet \vtmu_i^Y
                \Bigg),
    \end{multline*}
    (with $\vmu_i^W = \mu_i^W \vone$),
    and
    \(
        \gamma^{-1}
        \sqrt{ \frac{n}{p-k} }
        \to
        \gamma^{-1/2}.
    \)
\end{proof}

With Lemmas~\ref{L:eij-product-limits}~and~\ref{L:eij-product-scaled-limits},
we can simplify $f_n(z, \vx)$.  For convenience, we will define
$\mF_n (z)$ such that
\[
    \frac{1}{n}
    \mE_{n,12}
    \left(
        \mLambda_n
        -
        z
        \mI_{p-k}
    \right)^{-1}
    \mE_{n,12}^\trans
        =
            \gamma^{-1}
            m(\mu)
            \mI_k
            +
            \frac{1}{\sqrt{n}}
            \mF_n(z)
            +
            \ohP\left(
                \frac{1}{\sqrt{n}}
            \right)
\]
and $\mF_n(z) \tod \mF(z)$,
with $\mG_n(z), \mH_n(z), \mK_n, \mL_{n,1},$ and $\mL_{n,2}$ defined similarly.

\begin{lemma}
    If $[u,v] \subset \big( b(\gamma), \, \infty \big)$, then
    \(
        \mT_n( z )
            \toas
                \mT_0( z )
    \)
    uniformly on $[u,v]$, where
    \[
        \mT_0( z )
            =
                \frac{1}{ 1 + \gamma^{-1} m(z) }
                \mD^2 
                +
                \frac{1}{m(z)}
                \mI_k.
    \]
\end{lemma}

\begin{lemma}
    If $z > (1 + \gamma^{-1/2})^2$, then
    \[
        \mT_n( z ) 
            = 
                \mT_0( z )
                +
                \frac{1}{\sqrt{n}}
                \mT_{n,1}( z )
                +
                \ohP\left( \frac{1}{\sqrt{n}} \right),
    \]
    where
    \begin{equation*}
        \begin{split}
            \mT_{n,1}( z ) 
                &= \mL_{n,2}
                -
                \big( 1 + \gamma^{-1} m(z) \big)^{-2}
                    \cdot
                    \mD \mF_n(z) \mD \\
                &\quad\quad+ 
                \big( 1 + \gamma^{-1} m(z) \big)^{-1} \\
                    &\quad\quad\quad\cdot
                    \Big(
                        \sqrt{n} \left( \mD_n^2 - \mD^2 \right)
                        +
                        \mD \big( \mE_{n,11} - \mG_n(z) \big)
                        + 
                        \big( \mE_{n,11} - \mG_n(z) \big)^\trans \mD
                    \Big)  \\
                &\quad\quad-
                z \, \mH_n(z).
        \end{split}
    \end{equation*}
\end{lemma}
\begin{proof}
    First, we have
    \[
        \mS_{n,11} 
            = 
                \mD_n^2 + \mI_k
                + 
                \frac{1}{\sqrt{n}}
                \left(
                    \mD \mE_{n,11}
                    +
                    \mE_{n,11}^\trans \mD
                    +
                    \mL_{n,1}
                    +
                    \mL_{n,2}
                \right)
                +
                \ohP\left( \frac{1}{\sqrt{n}} \right).
    \]
    Next, we compute
    \begin{align*}
        \begin{split}
        \Bigg(
            \mI_k
            +
            \frac{1}{n}
            \mE_{n,12}
            &\Big(
                \mLambda_n - z \mI_{p-k}
            \Big)^{-1}
            \mE_{n,12}^\trans
        \Bigg)^{-1} \\
            &=
                \left(
                    \mI_k
                    +
                    \gamma^{-1}
                    m( z )
                    \mI_k
                    +
                    \frac{1}{\sqrt{n}}
                    \mF_n(z)
                    +
                    \ohP\Big( \frac{1}{\sqrt{n}} \Big)
                \right)^{-1} 
        \end{split} \\
            &=
                \big(1 + \gamma^{-1} m(z)\big)^{-1}
                \left(
                    \mI_k
                    +
                    \frac{1}{\sqrt{n}}
                    \big( 1 + \gamma^{-1} m(z) \big)^{-1}
                    \mF_{n}(z)
                    +
                    \ohP\Big( \frac{1}{\sqrt{n}} \Big)
                \right)^{-1} \\
            &=
                \big( 1 + \gamma^{-1} m(z) \big)^{-1} \mI_k
                -
                \frac{1}{\sqrt{n}}
                \big( 1 + \gamma^{-1} m(z) \big)^{-2}
                \mF_{n}(z)
                +
                \ohP\Big( \frac{1}{\sqrt{n}} \Big).              
    \end{align*}
    Using this, after some simplification we get
    \[
        \begin{split}
            \mS_{n,12} \Big( \mS_{n,22} &- z \mI_{p-k} \Big) \mS_{n,21} \\
                &=
                \gamma^{-1}
                \big( 1 + z \, m(z) \big) \mI_k
                +
                \frac{\gamma^{-1} m(z)}
                     { 1 + \gamma^{-1} m(z)} \mD_n^2 \\
                &\quad+
                \frac{1}{\sqrt{n}} \Bigg\{
                    \frac{1}{1 + \gamma^{-1} m(z)}
                        \left( 
                            \mD \mG_n(z) 
                            + \mG_n^\trans(z) \mD 
                        \right) \\
                    &\quad\qquad\qquad+
                    \frac{\gamma^{-1} m(z)}{1 + \gamma^{-1} m(z)}
                    \left( \mD \mE_{n,11} + \mE_{n,11}^\trans \mD \right) \\
                    &\quad\qquad\qquad+
                    \left(
                        \frac{1}{1 + \gamma^{-1} m(z)}
                    \right)^2
                    \mD \mF_n(z) \mD
                    +
                    \mL_{n,1}
                    +
                    z \, \mH_n( z ) \Bigg\} \\
                &\quad+
                \ohP\left( \frac{1}{\sqrt{n}} \right).
        \end{split}
    \]
    The equations for $\mT_0$ and $\mT_{n,1}$ follow.  We use the identity
    \[
        z
        \cdot
        \Big(
            1
            +
            \gamma^{-1} m(z)
        \Big)
        =
        -
        \frac{1}{m(z)}
        +
        (1 - \gamma^{-1})
    \]
    to simplify the form of $\mT_0$.
\end{proof}

\section{Solutions to the secular equation}

We will now study the solutions to $f_{n}(z, \vx) = 0$.  If $(z, \vx)$
is a solution, then so is $(z, \alpha \, \vx)$ for any scalar $\alpha$.  We restrict our attention to solutions with $\|\vx \|_2 = 1$.  We also impose a restriction on the sign of $\vx$, namely we require that the component with the largest magnitude is positive, i.e.
\(
    \max_i x_i = \max_i | x_i |.
\)

\subsection{Almost sure limits}

Given a solution of $f_0(z, \vx) = 0$, it is not hard to believe that there is a sequence of solutions $(z_n, \vx_n)$ converging to $z$ and $\vx$.  We make this precise in the next lemma.

\begin{lemma}
    If $f_0( z_0, \vx_0) = 0$, then with probability one there exists a 
    sequence $(z_n, \vx_n)$ satisfying $f_n(z_n, \vx_n) = 0$ such that
    $z_n \to z_0$ and $\vx_n \to \vx_0$.
\end{lemma}
\begin{proof}
    If $k = 1$, then the identifiability restrictions force $\vx_0 = 1$.  We
    define $g_n(z) = f_n(z, 1)$.  For $z > \lambda_{n,1}$, $g_n$ is 
    differentiable in a neighborhood of $z$, and $g'_n(z) > 0$.  The inverse 
    function theorem guarantees existence of a map
    $g_n^{-1} : \reals \to \reals$ defined in a neighborhood of $g_n(z)$ with
    $g_n^{-1}(g(z)) = z$
    
    this case for $z > \lambda_{n,1}$ we have that 
    $g_n(z) \equiv f_{n} (z, 1)$ is
    differentiable with $g'_n(z) > 0$.
    function in
    
    We the notation $\vx_{-k}$ to denote the vector
    $(x_1, x_2, \ldots, x_{k-1})$.
    Noting that $x_{k}$ is uniquely determined by $x_1, x_2, \ldots, x_{k-1}$, 
    define
    \[
        g_n( z, x_1, x_2, \ldots, x_{k-1})
            = f_n( z, x_1, x_2, \ldots, x_k).
    \]
    We have that on any compact set, 
    \[
        g_n( z, x_1, \ldots, x_{k-1} )
            = g_0( z, x_1, \ldots, x_{k-1} ) 
                + \epsilon_n( z, x_{1}, \ldots, x_{k-1} ),
    \]
    with $\epsilon_n(z, x_1, \ldots, x_{k-1}) \toas 0$.
    For $z > \lambda_{n,1}$, $g_n(z, x_1, \ldots, x_{k-1})$ 
    is differentiable in a neighborhood of $(z, x_1, \ldots, )$.
\end{proof}



\clearpage


\section{Limits of singular values}

We are finally ready to study the limits of the top eigenvalues of
$(1/n) X_n^T X_n$.  We have that
$(1/n) X_n^T X_n = \Psi_n \Psi_n^T + O \Lambda O^T$. Recall from
Lemma {L:secular} that the $j$th eigenvalue of $(1/n) X_n^T X_n$, say 
$\hat \mu_j$, is either equal to some $\lambda_i$ or else the matrix
$S_{n,1}(\hat \mu_j)$ has a unit eigenvalue; conversely all values of $z$ with
$S_{n,1}(z)$ having a unit eigenvalue are eigenvalues of $(1/n) X_n^T X_n$.
Lemma {L:lowrank-perturb}, in the Appendix, shows that  almost surely
$(1/n) X_n^T X_n$ has no nonzero eigenvalue equal to any $\lambda_i$.  Thus,
we can study the nonzero eigenvalues of $(1/n) X_n^T X_n$ by studying the
random matrix-valued function $S_{n,1}(z)$.

We denote by $\hat \mu_i$ the $i$th largest eigenvalue of $(1/n) X_n^T X_n$.

\subsection{Almost sure limits}\label{SS:value-limit}

Assume for now that $\gamma \leq 1$.  Denote by $\bar k$ the maximum index such that $\mu_{\bar k} > \sqrt{\gamma}$.  For $\varepsilon$ small enough, there are exactly $\bar k$ values of $z$ in  $(b + \varepsilon, \infty)$ for which $\bar S_1(z)$, the limit of $S_{n,1}(z)$ has a unit eigenvalue.  This is because 
\(
    \bar S_1(z) 
    = 
    \diag( r_1(z) (\mu_1 + \gamma), \ldots, r_1(z) (\mu_K + \gamma))
\)
and $r_1(z) < (\sqrt{\gamma} + \gamma)^{-1}$ on $(b, \infty)$.  A straightforward calculation shows that $r_1(z)$ has inverse
\[
    z(r_1)
    =
    \frac{1}{r_1}
    + 
    \frac{1}{1 - \gamma r_1}
\]
For $i \leq \bar K$, we define
\begin{align*}
    \tilde \mu_i
    &\equiv z\left( (\mu_i + \gamma)^{-1} \right) \\
    &= \left( \mu_i + 1 \right) 
       \left( 1 + \frac{\gamma}{\mu_i} \right)
\end{align*}
Due to the continuity of eigenvalues, for $i$ in this range, $\tilde \mu_i$ is
almost surely a limiting eigenvalue of $(1/n) X_n^T X_n$.  Thus for $i \leq K$,
$\hat \mu_i \toas \tilde \mu_i$.

Since $(1/n) O^T X_n^T X_n O - \Lambda$ is positive-semidefinite, the
Courant-Fischer min-max characterization of eigenvalues tells us that
$\hat \mu_i \geq \lambda_i$. Therefore, for $\bar k < i \leq k$,
\[
    b
    \leq
    \liminf_{n \to \infty} \hat \mu_i
    \leq
    \limsup_{n \to \infty} \hat \mu_i    
    \leq
    b + \varepsilon.
\]
Defining $\tilde \mu_i = b$ for $\bar k < i \leq k$, we can let
$\varepsilon \to 0$ to get that for $i$ in this range,
$\hat \mu_i \toas \tilde \mu_i$.

Now suppose that $\gamma > 1$.  We can use the work above to derive the limits of the top singular values since the nontrivial eigenvalues of $(1/n) X^T X$ and $\gamma (1/p) X X^T$ are the same.  If we replace $\gamma$ by $\gamma^{-1}$ and $\mu$ by $\mu \gamma^{-1}$, then we can apply the above theory.  The critical threshold is $\mu_i \gamma^{-1} > \gamma^{-1/2}$, i.e. $\mu_i > \sqrt{\gamma}$.  Above this threshold, the limit is given by $\gamma ( \mu \gamma^{-1} + 1)(1 + 1/\mu) = (\mu + 1)\left( 1 + \gamma/\mu \right)$.  Below the threshold, the limit is $\gamma (1 + \gamma^{-1/2})^2 = (1 + \gamma^{1/2})^2$.  We see that the formulas are the same as for $\gamma \leq 1$.

\subsection{Asymptotic normality}

Next, we turn our attention to the asymptotic distribution of 
$\sqrt{n} (\hat \mu_i - \tilde \mu_i)$.  Again, we start by assuming that $\gamma \leq 1$.  Let $\nu_i(z)$ denote the $i$th
eigenvalue of $S_{n,1}(z)$.  Since the off-diagonal elements of $S_{n,1}(z)$ are
of size $O_P(1/n^\half)$ and the diagonal elements are of size
$O(1) + O_P(1/n^\half)$, a standard perturbation argument
(which can be found, for example in \cite{anderson1963atp}) shows that
$\nu_i(z) = (S_{n,1}(z))_{ii} + O_P(1/n)$.
Since $S_{n,1}(z) \eqd \bar S_{n,1}(z) + (1/n^\half) S_1(z) + O_P(1/n)$, we
have that 
$\nu_i(z) \eqd r_1(z) (\mu_i + \gamma) + (1/n^\half) (S_1(z))_{ii} + O_P(1/n)$.
For $i \geq \bar k$, using the Taylor expansion 
$r_1(z) = r_1 (\tilde \mu_i) + r_1'(\tilde \mu_i) (z - \tilde \mu_i) + O(|z - \tilde \mu_i|^2)$
and the fact that $\nu_i (\hat \mu_i) = 1$, we have that
\[
    1 \eqd 1 
            + r_1' (\tilde \mu_i)(\hat \mu_i - \tilde \mu_i)(\mu_i + \gamma) 
            + (1/n^\half) (S_1(\hat \mu_i))_{ii}
            + O(|\hat \mu_i - \tilde \mu_i|^2) 
            + O_P(1/n)
\]
so that
\[
    \sqrt{n} (\hat \mu_i - \tilde \mu_i)
    \eqd
        -\frac{(S_1(\hat \mu_i))_{ii}}
              {r_1' (\tilde \mu_i) (\mu_i + \gamma)}
        + O(\sqrt{n}|\hat \mu_i - \tilde \mu_i|^2)       
        + O_P(1/n^\half).
\]
This in turn implies that $(\hat \mu_i - \tilde \mu_i) = O_P(1/n^\half)$.
Using a Taylor expansion of $S_1(z)$ about $\tilde \mu_i$, we get that jointly
for $i \leq \bar k$,
\[
    \sqrt{n} (\hat \mu_i - \tilde \mu_i)
        \tod -\frac{(S_1(\tilde \mu_i))_{ii}}
                   {r_1' (\tilde \mu_i) (\mu_i + \gamma)}.
\]
It is straightforward to show that 
\(
    r_2(\tilde \mu_i, \tilde \mu_j) 
    =
    \mu_i \mu_j 
    / 
    [ (\mu_i \mu_j - \gamma) (\mu_i + \gamma) (\mu_j + \gamma) ].
\)
Putting $i = j$, we have that
\(
    \left[ -r_1'(\tilde \mu_i ) \right]^{-1}
    =
    \left[ r_2(\tilde \mu_i, \tilde \mu_i) \right]^{-1}
    =
    \left( \mu_i + \gamma \right)^2
    \left( 1 - \gamma/\mu_i^2 \right)
\)
The previous section shows that jointly the unique elements
of $\{ S_1(\tilde \mu_i) \}_{i=1}^{\bar K}$ have a multivariate normal distribution.
Equations~{E:cov-S1-1}~and~{E:cov-S1-2} give
\begin{align*}
    \var [ (S_1(\tilde \mu_i))_{ii} ]
        &= \frac{1}{(\mu_i + \gamma)^2}
           \left[
               2 ( 2 \mu_i + \gamma + 1 )
               \left(
                   \frac{\mu_i^2}{\mu_i^2 - \gamma}
               \right)
               +
               \sigma_{ii}
           \right] \\
\intertext{and for $i \neq j$}
    \cov [ (S_1(\tilde \mu_i))_{ii},  (S_1(\tilde \mu_j))_{jj}]
        &= \frac{\sigma_{ij}}
                {(\mu_i + \gamma)(\mu_j + \gamma)}.
\end{align*}
Thus, the vector 
\(
    \sqrt{n} 
    (\hat \mu_1 - \tilde \mu_1, 
     \ldots, 
     \hat \mu_{\bar k} - \tilde \mu_{\bar k}
    )
\)
converges in distribution to a multivariate normal with the covariance
between elements $i$ and $j$ given by
\[
    \tilde \sigma_{ij}
    \equiv
    \sigma_{ij} 
    \frac{(\mu_i^2 - \gamma)(\mu_j^2 - \gamma)}{\mu_i^2 \mu_j^2}
    +
    \delta_{ij}
    2(2\mu_i + \gamma + 1)
    \left(
        \frac{\mu_i^2 - \gamma}{\mu_i^2}
    \right).
\]

When $\bar k < \mu_i \leq k$, to get asymptotic variance of
$\sqrt{n} (\hat \mu_i - \tilde \mu_i)$.  We must know finer asymptotics of
the leading eigenvalues $\lambda_1, \ldots, \lambda_k$.  Johnstone
\cite{johnstone2001dle} showed that $\lambda_1 = b + O_P(1/n^\frac{2}{3})$.
Soshnikov \cite{soshnikov2002nud} generalized this result to the top
$k$ eigenvalues, for fixed $k$.  Let $\varepsilon_n = \lambda_1 - b$  and
$\varepsilon_n' = b - \lambda_i$.  Denote by $Z_n$ the random variable
$b + 2 \varepsilon_n$. Since $Z_n > \lambda_1$ and $Z_n \toas b$,
the smallest eigenvalue of $S_{n,1} (Z_n)$ converges almost surely to a
value greater than $1$.  Thus, 
$\limsup_{n\to\infty} \hat \mu_i \leq \liminf Z_n$, and further
\[
    \limsup_{n\to\infty} \sqrt{n} \varepsilon_n'
    \leq
    \liminf_{n\to\infty} \sqrt{n} (\hat \mu_i - b)
    \leq
    \limsup_{n\to\infty} \sqrt{n} (\hat \mu_i - b)
    \leq
    \liminf_{n\to\infty} \sqrt{n} (2 \varepsilon_n).
\]
Therefore, $\sqrt{n} (\hat \mu_i - b) \toP 0$.

Now suppose that $\gamma > 1$.  Using the same transformation as the one at the end of Subsection~\ref{SS:value-limit}, it is straightforward to 
check that we get the same expression for the asymptotic covariance.

\section{Limits of singular vectors}

In this section we derive the limiting behavior of the singular vectors
of $(1/n) X_n^T X_n$, looking at the limits of $V^T \hat V$.

First suppose that $i \leq \bar k$.  Then $\hat \mu_i \toas \tilde \mu_i$ and
$S_{n,1}(\hat \mu_i) \toas \bar S_1 (\tilde \mu_i)$.  The $i$th eigenvalue
of $S_{n,1} (\hat \mu_i)$ converges almost surely to $1$ and the $i$th
eigenvector converges almost surely to $\be_i$.  This fact, 
Lemmas {L:secular}, Lemma {L:Sbar}, and equation {E:normY} 
tell us that $\hat \bv_i$, the normalized $i$th eigenvector of
$(1/n) X_n^T X_n$ almost surely has the same limit as
\[
    (\bar S_2(\tilde \mu_i))_{ii}^{-\half} R_1(\tilde \mu_i) \Psi_n \be_i.
\]
Combining this result with equation {E:corrY}, we get that
\begin{align*}
    V^T \hat \bv_i
        &\toas
        (\bar S_2(\tilde \mu_i))_{ii}^{-\half} \bar T(\tilde \mu_i) \be_i \\
        &= \left( r_2( \tilde \mu_i ) (\mu_i + \gamma )\right)^\half
           \left( r_1( \tilde \mu_i ) \mu_i^\half \right) \\
        &= \sqrt{
                \frac{ 1 - \gamma/\mu_i^2}
                     { 1 + \gamma/\mu_i  }
           }
\end{align*}

Next say that $\bar k < i \leq k$.  In this case, since 
$\lim_{z \to b^+} r_2(z) = \infty$ and $\lim_{z \to b^+} r_1(z) < \infty$, we must have that  $V^T \hat v_i \toas 0$.


%\section*{Acknowledgements}

%I would like to thank Persi Diaconis, Zongming Ma, Alexei Onatski, and
% Debashis Paul for their helpful communications and discussions. I would
% also like to thank Art Owen, my thesis advisor, for his support and 
% encouragement.
