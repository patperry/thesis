
\chapter{Properties of random projections}

We use this section to present some results about random projection
matrices.  We call a symmetric $n\times n$ matrix $\mP$ a projection 
if its eigenvalues are in the set $\{ 0, 1 \}$.  Any projection matrix of rank
$k$ can be decomposed as $\mP = \mV \mV^\trans$ for some $\mV$
satisfying $\mV^\trans \mV = \mI_k$.  We call the set 
\[
    \stiefel_{k}\left( \reals^n \right)
        =
        \left\{
            \mV \in \reals^{n \times k}
            :
            \mV^\trans \mV = \mI_k
        \right\}
        \subseteq
        \reals^{n \times k}
\]
the \emph{rank-$k$ Stiefel manifold of $\reals^n$}.  It is the set
of orthonormal $k$-frames in $\reals^n$.  Similarly, we call 
the set
\[
    \grassmann_k\left( \reals^n \right)
        =
        \left\{
            \mV \mV^\trans
            :
            \mV \in \stiefel_k \left( \reals^n \right)
        \right\}
        \subseteq
        \reals^{n \times n}
\]
the \emph{rank-$k$ Grassmannian of $\reals^n$}; this is the set of
rank $k$ projection matrices.  If
\(
    \left(
    \begin{matrix}
        \mV & \mbV
    \end{matrix}
    \right)
\)
is an $n\times n$ Haar-distributed orthogonal matrix and $\mV$ is
$n \times k$, then we say that $\mV$ is uniformly distributed
over $\stiefel_k( \reals^n )$ and that $\mV \mV^\trans$ is uniformly
distributed over $\grassmann_k(\reals^n)$.


\section{Uniformly distributed orthonormal $k$-frames}

We first present some results about a matrix $\mV$ distributed uniformly
over $\stiefel_k(\reals^n)$.  We denote this distribution by
\(
    \mV 
        \sim
            \Unif \big(
                \stiefel_k(\reals^n)
            \big).
\)

\subsection{Generating random elements}

The easiest way to generate a random element of $\stiefel_k(\reals^n)$ is
to let $\mZ$ to be an $n \times k$ matrix of \iid $\Normal( 0, \, 1)$
random variables, take the $QR$ decomposition $\mZ = \mQ \mR$, and let
$\mV$ be equal to the first $k$ columns of $\mQ$.  In practice, there is
a bias in the way standard $QR$ implementations choose the signs of the
columns of $\mQ$.  To get around this, we recommend using 
Algorithm~\ref{A:random-steifel}, below.

\begin{algorithm}
    \caption{\label{A:random-steifel}Generate a random orthonormal $k$-frame}
    \begin{enumerate}
        \item Draw $\mZ$, a random $n \times k$ matrix whose elements are 
              \iid $\Normal( 0, \, 1 )$ random variables.
              
        \item Compute $\mZ = \mQ \mR$, the $QR$-decomposition of $\mZ$.
              Set $\mQ_1$ to be the $n \times k$ matrix containing the
              first $k$ columns of $\mQ$.

        \item Draw $\mS$, a random $k \times k$ diagonal matrix with \iid 
              diagonal entries such that 
              \(
                  \Prob\left\{ S_{11} = -1 \right\}
                  =
                  \Prob\left\{ S_{11} = +1 \right\}
                  =
                  \frac{1}{2}.
              \)
        \item Return $\mV = \mQ_1 \mS$.
    \end{enumerate}
\end{algorithm}

\noindent 
This algorithm has a time complexity of $\Oh\left( n k^2 \right)$. Diaconis
and Shahshahani~\cite{diaconis1987sag} present an alternative approach called
the subgroup algorithm which can be used to generate $\mV$ as a product of 
$k$ Householder reflections.  Their algorithm has time complexity 
$\Oh\left( n k \right)$.  Mezzadri~\cite{mezzadri2007grm} gives a readable 
description of the subgroup algorithm.


\subsection{Mixed moments}

Since we can flip the sign of any row or column of $\mV$ and not change
its distribution, the mixed moments of the elements of $\mV$ vanish unless 
the number of elements from any row or column is even (counting multiplicity).  For example, $\E \left[ V_{11}^2 V_{21} \right] = 0$ since we can flip the sign of the second row of $\mV$ to get
\[
    V_{11}^2 V_{21} 
    \eqd
    V_{11}^2 (- V_{21})
    =
    -V_{11}^2 V_{21}.
\]
This argument does not apply to $V_{11} V_{12} V_{22} V_{21}$ or
$V_{11}^2 V_{22}^2$.


In the special case when $k = 1$, the vector
\(
    \left(
        V_{11}^2, V_{21}^2, \ldots, V_{n1}^2
    \right)
\)
is distributed as 
\(
    \Dirichlet\left( \frac{1}{2}, \frac{1}{2}, \ldots, \frac{1}{2} \right).
\)
This follows from the fact that if $Y_1, Y_2, \ldots, Y_n$ are independently distributed Gamma random variables and $Y_i$ has shape
$a_i$ and scale $s$, then with $S = \sum_{i=1}^n Y_i$, the vector
\(
    \left(
        \frac{Y_1}{S},
        \frac{Y_2}{S},
        \ldots
        \frac{Y_n}{S}
    \right)
\)
is distributed $\Dirichlet\left( a_1, a_2, \ldots, a_n \right)$.  Using
the standard formulas for Dirichlet variances and covariances, we get
the mixed moments up to fourth order.  They are summarized in the next
lemma.

\begin{lemma}\label{L:stiefel-1-moments}
    If $\mV \sim \Unif\big( \stiefel_1(\reals^n) \big)$, then
    \begin{subequations}
    \begin{align*}
        \E \left[ V_{11}^2 \right] 
            &= \frac{1}{n}, \\
        \E \left[ V_{11} V_{21} \right] 
            &= 0, \\
        \E \left[ V_{11}^4 \right] 
            &= \frac{3}{n (n + 2)}, \\
        \E \left[ V_{11}^2 V_{21}^2 \right] 
            &= \frac{1}{n (n + 2)}.
    \end{align*}
    \end{subequations}
    The odd mixed moments are all equal to zero.
\end{lemma}

Using Theorem~4 from Diaconis and Shahshahani~\cite{diaconis1994erm},
which gives the moments of the traces of Haar-distributed orthogonal
matrices, we can compute the mixed moments of $\mV$ for more general $k$.
Meckes~\cite{meckes2006ivs} gives an alternative derivation of these
results.

\begin{lemma}\label{L:stiefel-k-moments}
    If $\mV \sim \Unif\big( \stiefel_k(\reals^n) \big)$ and $k > 1$, then the
    nonzero mixed moments of the elements $\mV$ up to fourth order
    are defined by
    \begin{subequations}
    \begin{align*}
        \E \left[ V_{11}^2 \right] 
            &= 
                \frac{1}{ n }, \\
        \E \left[ V_{11}^4 \right] 
            &=
                \frac{3}{ n (n+2) }, \\
        \E \left[ V_{11}^2 V_{21}^2 \right] 
            &= 
                \frac{1}{ n (n+2) }, \\
        \E \left[ V_{11}^2 V_{22}^2 \right]
            &= 
                \frac{n+1}{ n (n-1) (n+2) }, \\
        \E \left[ V_{11} V_{12} V_{22} V_{21} \right] 
            &= 
                \frac{ -1 }{ n (n-1) (n+2) }.
    \end{align*}
    \end{subequations}
\end{lemma}
\begin{proof}
    The first three equations follow directly from the previous lemma.
    We can get the other moments from the moments of $\mO$, a Haar-distibuted
    $n \times n$ orthogonal matrix. For the fourth equation, we use that 
    \[
        \E \left[ \tr( \mO ) \right]^4
        =
        \sum_{r,s,t,u} \E \left[ O_{rr} O_{ss} O_{tt} O_{uu} \right].
    \]
    Only the terms with even powers of $O_{ii}$ are nonzero.  Thus, we have
    \begin{align*}
        \E \left[ \tr( \mO ) \right]^4
        &=   \binom{n}{1} \E \left[ O_{11}^4 \right]
           + \binom{n}{2} \binom{4}{2} \E \left[ O_{11}^2 O_{22}^2 \right] \\
        &=   n \E \left[ O_{11}^4 \right]
           + 3n(n-1) \E \left[ O_{11}^2 O_{22}^2 \right].
    \end{align*}
    Theorem~4 of Diaconis and Shahshahani \cite{diaconis1994erm} gives that 
    $\E \left[ \tr(\mO) \right]^4 = 3$.  Combined with
    Lemma~\ref{L:stiefel-1-moments}, we get that
    \begin{align*}
        \E \left[ O_{11}^2 O_{22}^2 \right]
        &= \frac{1}{3n(n-1)} 
            \left\{
                \E \left[ \tr(\mO) \right]^4
                - n \E \left[ O_{11}^4 \right]
            \right\} \\
        &= \frac{1}{3n(n-1)} 
           \left\{
                3 - n \frac{3}{n(n+1)}
           \right\} \\
        &= \frac{n+1}{n(n-1)(n+2)}.
    \end{align*}
    
    For the last equation, we use $\E\left[ \tr(\mO^4) \right]$.  We have that 
    \[
        (\mO^4)_{ij} = \sum_{r,s,t} O_{ir} O_{rs} O_{st} O_{tj}.
    \]
    We would like to compute $\E\left[ (\mO^4)_{11} \right]$.  Note that
    unless $r=s=t=1$, there are only three situations when
    $\E \left[ O_{1r} O_{rs} O_{st} O_{t1} \right] \neq 0$.   Two of them are 
    demonstrated visually by the configurations
    \[
        \begin{matrix}
            \phantom{0} &
            \begin{matrix} 
                1 & \phantom{\cdots} & s \phantom{\cdots}
            \end{matrix} \\
            \begin{matrix}
                1 \\
                \phantom{\vdots}  \\
                s \\
                \phantom{\vdots}
            \end{matrix}&
            \left(
            \begin{matrix}
                O_{1r} & \cdots & O_{rs} & \cdots \\
                \vdots  & \ddots & \vdots  & \\
                O_{t1} & \cdots & O_{st} & \cdots \\
                \vdots  &        & \vdots  &
            \end{matrix}
            \right) \\
            \phantom{0} &
            r=1, s=t, s \neq 1
        \end{matrix}
        \quad\quad\text{and}\quad\quad
        \begin{matrix}
            \phantom{0} &
            \begin{matrix} 
                1 & \phantom{\cdots} & r \phantom{\cdots}
            \end{matrix} \\
            \begin{matrix}
                1 \\
                \phantom{\vdots}  \\
                r \\
                \phantom{\vdots}
            \end{matrix}&
            \left(
            \begin{matrix}
                O_{t1} & \cdots & O_{1r} & \cdots \\
                \vdots  & \ddots & \vdots  & \\
                O_{st} & \cdots & O_{rs} & \cdots \\
                \vdots  &        & \vdots  &
            \end{matrix}
            \right) \\
            \phantom{0} &
            t=1, r=s, r \neq 1
        \end{matrix}.
    \]
    The other nonzero term is when $s=1$, $r=t$, and $r\neq1$, so
    that $O_{1r} O_{rs} O_{st} O_{t1} = O_{1r}^2 O_{r1}^2$.
    In all other configurations there is a row or a column that only
    contains one of $\{ O_{1r}, \, O_{rs}, \, O_{st}, \, O_{t1} \}$.  Since 
    we can multiply a row or a column of $\mO$ by $-1$ and not change the
    distribution of $\mO$, for this choice of $r$, $s$, and $t$ we must have
    $O_{1r} O_{rs} O_{st} O_{t1} \overset{d}{=} -O_{1r} O_{rs} O_{st} O_{t1}$.
    This in turn implies that 
    $\E\left[O_{1r} O_{rs} O_{st} O_{t1} \right] = 0$.
    With these combinatorics in mind, we have that
    \begin{align*}
        \E \! \left[ (\mO^4)_{11} \right]
        \!&=\!
            \sum_{s \neq 1} \E \left[ O_{11} O_{1s} O_{ss} O_{s1} \right]
            +
            \sum_{r \neq 1} \E \left[ O_{1r} O_{rr} O_{r1} O_{11} \right]
            +
            \sum_{r \neq 1} \E \left[ O_{1r}^2 O_{r1}^2 \right]
            + 
            \E \left[ O_{11}^4 \right] \\
        &=
            2(n-1) \E \left[ O_{11} O_{12} O_{22} O_{21} \right]
            + (n-1) \E \left[ O_{12}^2 O_{21}^2 \right]
            + \E \left[ O_{11}^4 \right] \\
        &=
            2(n-1) \E \left[ O_{11} O_{12} O_{22} O_{21} \right]
            + (n-1) \E \left[ O_{11}^2 O_{22}^2 \right]
            + \E \left[ O_{11}^4 \right]
    \end{align*}
    Again applying Theorem~4 of \cite{diaconis1994erm}, we have that 
    $\E \left[ \tr (\mO^4) \right] = 1$.  Combined with 
    Lemma~\ref{L:stiefel-1-moments}, we
    have
    \begin{align*}
        \E \left[ O_{11} O_{12} O_{22} O_{21} \right]
        &= 
        \frac{1}{2(n-1)}
        \left\{
            \frac{1}{n} \E \left[ \tr (\mO^4) \right]
            - \E\left[ O_{11}^4 \right]
            - (n-1) \E \left[ O_{11}^2 O_{22}^2 \right]
        \right\} \\
        &=
        \frac{1}{2(n-1)}
        \left\{
            \frac{1}{n}
            - \frac{3}{n(n+2)}
            - (n-1) \frac{n+1}{n(n-1)(n+2)}
        \right\} \\
        &= 
        \frac{-1}{n(n-1)(n+2)}. \qedhere
    \end{align*} 
\end{proof}


\section{Uniformly distributed projections}

When $\mP \in \reals^{n \times n}$ is chosen uniformly over the set of rank-$k$ $n\times n$ projection matrices, we say
\(
    \mP
        \sim
            \Unif \big(
                \grassmann_k(\reals^n)
            \big).
\)
With the results of the previous section, we can derive the moments of
random projection matrices.


\begin{lemma}\label{L:grassman-moments-1}
    If $\mP \sim \Unif \big( \grassmann_k(\reals^n) \big)$, then
    \begin{align*}
        \E \left[ \mP \right] &= \frac{k}{n} \mI_n.
    \end{align*}
\end{lemma}
\begin{proof}
    Write $\mP = \mV \mV^\trans$, where 
    \(
        \mV \sim \Unif \big( \stiefel_k( \reals^n ) \big)
    \).
    For $1 \leq i,j \leq n$ we have
    \[
        P_{ij}
            =
                \sum_{r=1}^k V_{ir} V_{jr},
    \]
    so
    \[
        \E \left[ P_{ij} \right] 
        =
        \begin{cases}
            k \, \E \left[ V_{11}^2 \right], &\text{when $i = j$,} \\
            k \, \E \left[ V_{11} V_{21} \right], &\text{otherwise.}
        \end{cases}
    \]
    The result now follows from Lemma~\ref{L:stiefel-1-moments}.
\end{proof}

\begin{lemma}\label{L:grassman-moments-2}
    Let $\mP \sim \Unif \big( \grassmann_k(\reals^n) \big)$.  If
    $1 \leq i,j,i',j' \leq n$, then
    \begin{equation*}
        \begin{split}
            \cov \big[ P_{ij}, P_{i'j'} \big]
                &=
                    \frac{1}{(n-1)(n+2)}
                    \left(
                        \frac{k}{n}
                    \right)
                    \left(
                        1 - \frac{k}{n}
                    \right) \\
                &\qquad\cdot
                    \Big(
                          n \, \delta_{(i,j ) = (i',j')}
                        + n \, \delta_{(i,j ) = (j',i')}
                        - 2 \, \delta_{(i,i') = (j ,j')}
                    \Big).
        \end{split}
    \end{equation*}
    This gives us that aside from the obvious symmetry ($P_{ij} = P_{ji}$), 
    the off-diagonal elements of $\mP$ are uncorrelated with each other and 
    with the diagonal elements.
\end{lemma}
\begin{proof}
    We need to perform six computations.  As before, we use the representation
    $\mP = \mV \mV^\trans$, where 
    \(
        \mV \sim \Unif\big( \stiefel_k(\reals^n) \big).
    \)
    We have
    \begin{align*}
        \E \left[ P_{11}^2 \right]
        &= \E \bigg[ \Big( 
                         \sum_{i=1}^k V_{1i}^2
                     \Big)^2
              \bigg] \\
        &= \E \bigg[
                  \sum_{i=1}^k V_{1i}^4
                  + \sum_{i\neq j} V_{1i}^2 V_{1j}^2
              \bigg] \\
        &= k \cdot \frac{3}{ n \, (n+2) } 
           + k \, (k-1) \cdot \frac{1}{ n \, (n+2) } \\
        &= \frac{ k \, (k+2) }{ n \,(n+2) } \\
        &= \frac{2}{ n+2}
           \left(\frac{k}{n}\right) \left( 1 - \frac{k}{n} \right)
           + \left( \frac{k}{n} \right)^2,
    \end{align*}
    which gives us that 
    \[
        \var \left[ P_{11} \right]
        = \frac{2}{ n+2 }
           \left(\frac{k}{n}\right) \left( 1 - \frac{k}{n} \right).
    \]
    Next,
    \begin{align*}
        \E \left[ P_{12}^2 \right]
        &= \E \bigg[ \Big( 
                         \sum_{i=1}^k V_{1i} V_{2i}
                     \Big)^2
              \bigg] \\
        &= \E \bigg[ 
                  \sum_{i=1}^k V_{1i}^2 V_{2i}^2
                  + \sum_{i\neq j} 
                        V_{1i} V_{2i} V_{1j} V_{2j}
              \bigg] \\
        &= k \cdot \frac{ 1 }{ n \, (n+2) }
           + k \, (k-1) \cdot \frac{-1}{ n \, (n-1) (n+2) } \\
        &= \frac{n}{ (n-1) (n+2) } 
           \left(\frac{k}{n}\right) \left( 1 - \frac{k}{n} \right),
    \end{align*}
    so that
    \[
        \var \left[ P_{12} \right]
        = \frac{n}{ (n-1) (n+2) } 
           \left(\frac{k}{n}\right) \left( 1 - \frac{k}{n} \right).
    \]
    Also,
    \begin{align*}
        \E \left[ P_{11} P_{22} \right]
        &= \E \bigg[
                \Big( \sum_{i=1}^k V_{1i}^2 \Big)
                \Big( \sum_{j=1}^k V_{2j}^2 \Big)
              \bigg] \\
        &= \E \bigg[
                  \sum_{i=1}^k V_{1i}^2 V_{2i}^2
                  + \sum_{i=1}^k
                    \sum_{j\neq i} V_{1i}^2 V_{2j}^2
              \bigg] \\
        &= k \cdot \frac{1}{ n \, (n+2) }
           + k \, (k-1) \cdot \frac{ n+1 }{ n \, (n-1) (n+2) } \\
        &= \frac{k}{ n \, (n+2) }
           \left( 1 + (n+1) \frac{k-1}{n-1} \right) \\
        &= \frac{-2}{(n-1)(n+2)}
           \left(\frac{k}{n}\right) \left( 1 - \frac{k}{n} \right)
           + \left( \frac{k}{n} \right)^2,
    \end{align*}
    so that
    \[
        \cov \left[ P_{11}, P_{22} \right]
        = \frac{-2}{ (n-1) (n+2) }
           \left(\frac{k}{n}\right) \left( 1 - \frac{k}{n} \right).
    \]
    Since $\mP$ is symmetric, we have
    \[
        \cov \left[ P_{12}, P_{21} \right]
        = \var \left[ P_{12} \right].
    \]
    The other covariances are all zero.  This is because
    \begin{align*}
        \E \left[ P_{11} P_{12} \right]
        &= \E \bigg[ 
                  \sum_{i,j} V_{1i}^2 \, V_{1j}  V_{2j} 
              \bigg], \\
        \E \left[ P_{12} P_{23} \right]
        &= \E \bigg[ 
                \sum_{i,j} V_{1i} V_{2i} V_{2j} V_{3j}
              \bigg], \\
    \intertext{and}
        \E \left[ P_{12} P_{34} \right]
        &= \E \bigg[ 
                \sum_{i,j} V_{1i} V_{2i} V_{3j} V_{4j}
              \bigg].
    \end{align*}
    Each term in these sums has an element that appears only once in a row.
    Thus, the expectations are all $0$.
\end{proof}


\section{The Johnson-Lindenstrauss lemma}

As an application of the calculations in this section, we can derive the
Johnson-Lindenstrauss lemma.

\begin{lemma}
    If $\vx, \vy \in \reals^n$ and 
    \(
        \mP \sim \Unif\big( \grassmann_k (\reals^n) \big),
    \)
    then
    \[
        \E \left[ \vx^\trans \mP \vy \right] 
            = \frac{k}{n} \, \vx^\trans \vy,
    \]
    and
    \[
        \var \left[ \vx^\trans \mP \vy \right] 
            =
                \left( \frac{k}{n} \right)
                \left( 1 - \frac{k}{n} \right)
                \cdot 
                \frac{ n \, ( \vx^\trans \vx )( \vy^\trans \vy )
                       + (n-2) ( \vx^\trans \vy )^2 }
                     { (n-1) (n+2) }.
    \]
    In particular, this implies that
    \[
        \E \left[ \vx^\trans \mP \vx \right] 
        = 
        \frac{k}{n} \vx^\trans \vx
    \]
    and
    \[
        \var \left[ \vx^\trans \mP \vx \right]
        =
        \frac{ 2 }{ n+2 }
        \left( \frac{k}{n} \right)
        \left( 1 - \frac{k}{n} \right)
        \left( \vx^\trans \vx \right)^2.
    \]
\end{lemma}
\begin{proof}
    We have
    \[
        \vx^\trans \mP \vy
            = \sum_{i,j} x_i \, P_{ij} \, y_j.
    \]
    This gives us that
    \begin{align*}
        \E \left[ \vx^\trans \mP \vy \right]
            &= \sum_{i,j} \frac{k}{n} \, x_i \, y_j \, \delta_{ij} \\
            &= \frac{k}{n} \, \vx^\trans \vy
    \end{align*}
    and
    \begin{align*}
        \begin{split}
        \var \left[ \vx^\trans \mP \vy \right]
            &= 
               \sum_{i=1}^n
                    x_i^2 y_i^2 \var \left[ P_{ii} \right]
               +
               \frac{1}{2}
               \sum_{i \neq j} 
                    (x_i y_j + x_j y_i)^2 
                    \var \left[ P_{ij} \right] \\
            &\qquad+
               \sum_{i \neq j}
                    x_i x_j y_i y_j
                    \cov \left[ P_{ii}, P_{jj} \right]
        \end{split} \\
        \begin{split}
            &=
                \left( \frac{k}{n} \right)
                \left( 1 - \frac{k}{n} \right)
                \cdot \bigg\{
                    \frac{2}{n+2}
                    \sum_{i=1}^n x_i^2 y_i^2 \\
            &\qquad\qquad\qquad\qquad\quad+
                \frac{n}{ 2 \, (n-1) (n+2) }
                \sum_{i\neq j} (x_i y_j + x_j y_i)^2 \\
            &\qquad\qquad\qquad\qquad\quad+
                \frac{-2}{ (n-1) (n+2) }
                \sum_{i \neq j} x_i x_j y_i y_j \bigg\}
        \end{split} \\
        \begin{split}
            &=
                \left( \frac{k}{n} \right)
                \left( 1 - \frac{k}{n} \right)
                \cdot \bigg\{
                    \frac{n}{ (n-1) (n+2) } 
                    ( \vx^\trans \vx )( \vy^\trans \vy ) \\
            &\qquad\qquad\qquad\qquad\quad+
                \frac{n-2}{ (n-1) (n+2) } ( \vx^\trans \vy )^2 \bigg\}
        \end{split}
    \end{align*}
\end{proof}