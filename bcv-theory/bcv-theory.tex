\chapter[Optimal leave-out size for BCV]{Optimal leave-out size for bi-cross-validation}
\label{C:optimal-leave-out}

In this chapter we will determine the optimal leave out-size for Gabriel-style
cross-validation of an SVD, also known as bi-cross-validation (BCV), along
with proving a weak form of consistency.  In Chapter~\ref{C:intrinsic-rank},
we rigorously defined the rank estimation problem, and in 
Chapter~\ref{C:cvul} we introduced Gabriel-style cross validation.  Here, we 
provide theoretic justification for Gabriel-style CV.

First, a quick review of the problem.  We are given $\mX$, and $n \times p$ matrix generating by a ``signal-plus-noise'' process, 
\[
    \mX = \sqrt{n} \, \mU \mD \mV^\trans + \mE.
\]
Here, $\mU \in \reals^{n \times k_0}$, $\mD \in \reals^{k_0 \times k_0}$,
$\mV \in \reals^{p \times k_0}$, and $\mE \in \reals^{n \times p}$.  The first
term, $\sqrt{n} \mU \mD \mV^\trans$, is the low-rank ``signal'' part.  We call 
$\mU$ and $\mV$ the matrices of left and right factors, respectively.  They 
are normalized so that $\mU^\trans \mU = \mV^\trans \mV = \mI_{k_0}$.  The 
factors ``strengths'' are given in $\mD$, a diagonal matrix of the form
\(
    \mD =
        \diag( d_1, d_2, \ldots, d_{k_0}),
\)
with $d_1 \geq d_2 \geq \cdots \geq d_{k_0} \geq 0$.  Also, typically $k_0$ is 
much smaller than $n$ and $p$.  Lastly, $\mE$ consists of ``noise''.  Although more general types of noise are possible, for simplicity we will assume that $\mE$ is independent of $\mU$, $\mD$, and $\mV$.  We think 
of the signal part as the important part of $\mX$, and the noise part is 
inherently uninteresting.

The rank estimation problem is find the optimal number of terms of the SVD of $\mX$ to keep to estimate the signal part.  We let 
$\mX = \sqrt{n} \, \mhU \mhD \mhV^\trans$ be the SVD of $\mX$, where 
$\mhU \in \reals^{n \times n \wedge p}$ and 
$\mhV \in \reals^{p \times n \wedge p}$ have orthonormal columns, and
\(
    \mhD = \diag(
        \hd_1, \hd_2, \ldots, \hd_{n \wedge p}
    )
\)
with $\hd_1 \geq \hd_2 \geq \cdots \geq \hd_{n \wedge p}$.  For $0 \leq k \leq n \wedge p$, we define
\(
    \mhD(k) = \diag(
        \hd_1, \hd_2, \ldots, \hd_k, 0, 0, \ldots, 0
    )
\)
so that $\mhX(k) \equiv \sqrt{n} \, \mhU \mhD(k) \mhV^\trans$ is the SVD of $\mX$ truncated to $k$ terms.  The model error with respect to Frobenius loss is given by
\[
    \ME(k) = \frac{1}{n \, p}
             \| \sqrt{n} \, \mU \mD \mV^\trans - \mhX(k) \|_\Frob^2
\]
The optimal rank is defined with respect to this criterion as
\[
    k^\ast = \argmin_k \ME(k).
\]
The problem we consider is how to estimate $\ME(k)$ or $k^\ast$.

Closely related to model error is the \emph{prediction error}.  For prediction error, we conjure a noise matrix $\mE'$ with the same distribution as $\mE$ and let $\mX' = \sqrt{n} \mU \mD \mV^\trans + \mE'$.  The prediction error is defined as
\[
    \PE(k) \equiv \frac{1}{n \, p} \E[ \mX' - \mhX(k) ],
\]
which can be expressed as
\[
    \PE(k) = \E[ \ME(k) ] + \frac{1}{n \, p} \E \| \mE \|_\Frob^2.
\]
The minimizer of $\PE$ is the same as the minimizer of $\E[ \ME(k) ]$,
and one can get an estimate of $\ME$ a $\PE$ estimate by subtracting an estimate of the noise level.

The previous chapter suggests using Gabriel-style cross-validation for estimating the optimal rank.  Owen \& Perry~\cite{owen2009bi} call this procedure bi-cross-validation (BCV).  For fold $(i,j)$ of BCV, we permute
the rows of $\mX$ with matrices $\mP^{(i)}$ and $\mQ^{(j)}$, then partition the result into four blocks as
\[
    \mP^{(i)\trans} \mX \mQ^{(j)}
        =
        \left(
        \begin{matrix}
            \mX_{11} & \mX_{12} \\
            \mX_{21} & \mX_{22}
        \end{matrix}
        \right).
\]
We take the SVD of the upper-left block and evaluate its predictive 
performance on the lower-right block.  If 
$\mX_{11} = \sqrt{n} \mhU_1 \mhD_1 \mhV^\trans$ is the SVD of $\mX_{11}$ and
$\mhX_{11}(k) = \sqrt{n} \mhU_1 \mhD_1(k) \mhV^\trans$ is its truncation to 
$k$ terms, then the BCV estimate of prediction error from this fold is given 
by
\[
    \widehat{\PE}\big(k ; i, j \big)
        =
            \frac{1}{n_2 \, p_2}
            \| \mX_{22} - \mX_{21} \mhX_{11}(k)^+ \mX_{12} \|_\Frob^2,
\]
where $^+$ denotes pseudo-inverse and $\mX_{22}$ has dimensions $n_2 \times p_2$.  For $(K,L)$-fold BCV, the final estimate is the average over all folds:
\[
    \widehat{\PE}(k)
        =
        \frac{1}{K L}
        \sum_{i=1}^K
        \sum_{j=1}^L
            \widehat{\PE}\big(k ; i, j \big).
\]
From $\widehat{\PE}(k)$ we can get an estimate of the optimal rank as
$\hat k = \argmin_k \widehat{\PE}(k)$.

In this chapter, we give a theoretical analysis of $\widehat{\PE}(k)$.  This allows us to determine the bias inherent in $\widehat{\PE}(k)$ and its consistency properties for estimating $k^\ast$, along with guidance for choosing the number of folds ($K$ and $L$).


\section{Assumptions and notation}

The theory becomes easier if we work in an asymptotic framework.  For that,
we introduce a sequence of data matrices indexed by $n$:
\[
    \mX_{n} = \sqrt{n} \, \mU_n \mD_n \mV_n^\trans + \mE_n.
\]
Here, $\mX_n \in \reals^{n \times p}$ with $p = p(n)$ and 
$\frac{n}{p} \to \gamma \in (0,\infty)$.  Even though the dimensions of $\mX_n$ grow, we assume that the number of factors is fixed at $k_0$.  The full set of assumptions is as follows:

\begin{assumption}\label{A:optimal-leave-out-shapes}
    We have a sequence of random matrices $\mX_n \in \reals^{n \times p}$ 
    with $n \to \infty$ and $p = p(n)$ also going to infinity.  Their ratio
    converges to a fixed constant $\gamma \in (0, \infty)$ as 
    $\frac{n}{p} = \gamma + \oh\left( \frac{1}{\sqrt{n}} \right)$.
\end{assumption}

\begin{assumption}
    The matrix $\mX_n$ is generated as 
    $\mX_n = \sqrt{n} \, \mU_n \mD_n \mV_n^\trans + \mE_n$.  Here,
    $\mU_n \in \reals^{n \times k_0}$, 
    $\mD_n \in \reals^{k_0 \times k_0}$,
    $\mV_n \in \reals^{p \times k_0}$, and $\mE_n \in \reals^{n \times p}$.
    The number of factors, $k_0$, is fixed.
\end{assumption}

\begin{assumption}
    The matrices of left and right factors, $\mU_n$ and $\mV_n$, have 
    orthonormal columns, i.e. 
    $\mU_n^\trans \mU_n = \mV_n^\trans \mV_n = \mI_{k_0}$.
    Their columns are denoted by
    $\vu_{n,1}, \vu_{n,2}, \ldots, \vu_{n,k_0}$ and
    $\vv_{n,1}, \vu_{n,2}, \ldots, \vv_{n,k_0}$, respectively.
\end{assumption}

\begin{assumption}
    The matrix of factor strengths is diagonal:
    \[
        \mD_n = \diag( d_{n,1}, d_{n,2}, \ldots, d_{n,k_0}).  
    \]
    The strengths
    converge as $d_{n,i}^2 \toas \mu_i$ and 
    $d_{n,i}^2 - \mu_i = \OhP\left( \frac{1}{\sqrt{n}} \right)$, strictly
    ordered as
    \(
        \mu_1 > \mu_2 > \cdots > \mu_{k_0} > 0.
    \)
\end{assumption}

\begin{assumption}\label{A:optimal-leave-out-noise}
    The noise matrix $\mE_n$ is independent of $\mU_n$, $\mD_n$, and $\mV_n$.
    Its elements are \iid with 
    \(
        E_{n,11} \sim \Normal( 0, \, \sigma^2 ).
    \)
\end{assumption}

\noindent
These assumptions are stanford for latent factor models.

We can apply the work of allow us to apply the work of
Chapter~\ref{C:intrinsic-rank} to get the behavior of the model error.
    We
let $\mX_n = \sqrt{n} \, \mhU_n \mhD_n \mhV_n^\trans$ be the SVD of $\mX_n$ and let $\mhX_n(k) = \sqrt{n} \, \mhU_n \mhD_n(k) \mhV_n^\trans$ be its truncation to $k$ terms.  Then the model error is
\[
    {\ME}_n (k)
        = 
            \frac{1}{n p} 
            \big\| 
                \sqrt{n} \mU_n \mD_n \mV_n^\trans 
                - 
                \mhX_n(k)
            \big\|_\Frob^2.
\]
With Assumtions~\ref{A:optimal-leave-out-shapes}--\ref{A:optimal-leave-out-noise}, we can apply Proposition~\ref{P:frob-loss-behavior} to get that for fixed
$k$ as $n \to \infty$,
\[
    p \cdot {\ME}_n (k)
        \toas
        \sum_{i=1}^{k}
            \alpha_i \mu_i
        +
        \sum_{i=k+1}^{k_0}
            \mu_i
        +
        \sigma^2
        \left(
            1 + \frac{1}{\sqrt{\gamma}}
        \right)^2
        \cdot
        (k - k_0)_+,
\]
where
\[
    \alpha_i 
    =
    \begin{cases}
        \frac{\sigma^2}{\gamma \mu_i^2}
                \big(
                    3 \sigma^2 + (\gamma+1) \mu_i
                \big)
            &\text{if $\mu_i > \frac{\sigma^2}{\sqrt{\gamma}}$,} \\
        1 
        + 
        \frac{\sigma^2}{\mu_i}
        \left(
            1
            +
            \frac{1}{\sqrt{\gamma}}
        \right)^2
            &\text{otherwise.}
    \end{cases}
\]
Defining $k_n^\ast$ as the minimizer of ${\ME}_n(k)$, we also get that
\[
    k_n^\ast
        \toas
            \max \left\{ i : \mu_i > \mu_\text{crit} \right\},
\]
where
\[
    \mu_\text{crit}
    \equiv
    \sigma^2
    \left(
    \frac{1 + \gamma^{-1}}{2}
    +
    \sqrt{ 
        \left( \frac{1 + \gamma^{-1} }{2} \right)^2
        +
        \frac{3}{\gamma}
    }
    \right),
\]
provided no $\mu_i$ is exactly eqaul to $\mu_\text{crit}$.  We therefore know
how ${\ME}_n(k)$ and its minimizer behave.

To study the bi-cross-validation estimate of prediction error, we need to 
introduce some more notation.  As we are only analyzing first-order 
behavior, we can restrict our analysis to the prediction error estimate from a 
single fold.  We let $\mP_n \in \reals^{n \times n}$ and 
$\mQ_n \in \reals^{p \times p}$ be permutation matrices for the fold, partitioned as
\(
    \mP_n 
        = 
        \left( 
        \begin{matrix}
            \mP_{n,1} & \mP_{n,2}
        \end{matrix}
        \right)
\)
and
\(
    \mQ_n
        =
        \left(
        \begin{matrix}
            \mQ_{n,1} & \mQ_{n,2}
        \end{matrix}
        \right),
\)
with $\mP_{n,1} \in \reals^{n \times n_1}$, $\mP_{n,2} \in \reals^{n \times n_2}$, $\mQ_{n,1} \in \reals^{p \times p_1}$, and $\mQ_{n,2} \in \reals^{p \times p_2}$.  Note that $n = n_1 + n_2$ and that $p = p_1 + p_2$.  We define
$\mX_{n,ij} = \mP_{n,i}^\trans \mX \mQ_{n,j}$,
$\mE_{n,ij} = \mP_{n,i}^\trans \mE \mQ_{n,j}$,
$\mU_{n,i} = \mP_{n,i}^\trans \mU_n$, and
$\mV_{n,j} = \mQ_{n,j}^\trans \mV_n$.  Then in block form,
\begin{align*}
    \mP_n^\trans \mX_n \mQ_n
        &=
            \left(
            \begin{matrix}
                \mX_{n,11} & \mX_{n,12} \\
                \mX_{n,21} & \mX_{n,22}
            \end{matrix}
            \right) \\
        &=
            \sqrt{n}
            \left(
            \begin{matrix}
                \mU_{n,1} \mD_n \mV_{n,1}^\trans &
                    \mU_{n,1} \mD_n \mV_{n,2}^\trans \\
                \mU_{n,2} \mD_n \mV_{n,1}^\trans &
                    \mU_{n,2} \mD_n \mV_{n,2}^\trans
            \end{matrix}
            \right)
            +
            \left(
            \begin{matrix}
                \mE_{n,11} & \mE_{n,12} \\
                \mE_{n,21} & \mE_{n,22}
            \end{matrix}
            \right)
\end{align*}
This is the starting point of our analysis.

Next, we describe the estimate of prediction error.  We let
$\mX_{n,11} = \sqrt{n} \, \mhU_{n,1} \mhD_{n,1} \mhV_{n,1}^\trans$ be the
SVD of $\mX_{n,11}$.  Here,
\begin{align*}
    \mhU_{n,1}
        &=
            \left(
            \begin{matrix}
                \vhu_{n,1}^{(1)} & 
                \vhu_{n,2}^{(1)} & 
                \cdots & 
                \vhu_{n,n_1 \wedge p_1}^{(1)}
            \end{matrix}
            \right), \\
    \mhV_{n,1}
        &=
            \left(
            \begin{matrix}
                \vhv_{n,1}^{(1)} & 
                \vhv_{n,2}^{(1)} & 
                \cdots & 
                \vhv_{n,n_1 \wedge p_1}^{(1)}
            \end{matrix}
            \right),
\intertext{and}
    \mhD_{n,1}
        &=
            \diag \left(
                \hd_{n,1}^{(1)},
                \hd_{n,2}^{(1)},
                \ldots,
                \hd_{n,n_1 \wedge p_1}^{(1)}
            \right).
\end{align*}
For convenience, we define $\hmu_{n,i}^{(1)} = \big(\hd_{n,i}^{(1)}\big)^2$.
For $0 \leq k \leq n_1 \wedge p_1$, we let
\[
    \mhD_{n,1}(k)
        =
            \diag \left(
                \hd_{n,1}^{(1)},
                \hd_{n,2}^{(1)},
                \ldots,
                \hd_{n,k}^{(1)},                
                0,
                0,
                \ldots,
                0
            \right)
\]
so that 
\(
    \mhX_{n,11} (k)
        \equiv \sqrt{n} \, \mhU_{n,1} \mhD_{n,1}(k) \mhV_{n,1}^\trans
\)
is the SVD of $\mhX$ truncated to $k$ terms.  It has pseudo-inverse
\(
    \mhX_{n,11}(k)^+
        =
            \frac{1}{\sqrt{n}} \, 
            \mhV_{n,1} \mhD_{n,1}(k)^+ \mhU_{n,1}^\trans.
\)
The BCV rank-$k$ prediction of $\mX_{22}$ is
\begin{align*}
    \mhX_{n,22}(k)
        &=
            \mX_{n,21} \mhX_{n,11}^{+}(k) \mX_{n,12} \\
        &=
            \big(
                \sqrt{n} \, \mU_{n,2} \mD_n \mV_{n,1}^\trans 
                + \mE_{n,21}
            \big)
            \big(
                \mhX_{n,11}(k)^+
            \big)
            \big(
                \sqrt{n} \, \mU_{n,1} \mD_n \mV_{n,2}^\trans 
                + \mE_{n,12}
            \big) \\
    \begin{split}
        &= 
            \sqrt{n} \, \mU_{n,2} \mD_n \mV_{n,1}^\trans
                \mhV_{n,1} \mhD_{n,1}(k)^+ \mhU_{n,1}^\trans
                \mU_{n,1} \mD_n \mV_{n,2}^\trans \\
        &\quad+
            \mU_{n,2} \mD_n \mV_{n,1}^\trans
                \mhV_{n,1} \mhD_{n,1}(k)^+ \mhU_{n,1}^\trans
                \mE_{n,12} \\
        &\quad+
            \mE_{n,21}
                \mhV_{n,1} \mhD_{n,1}(k)^+ \mhU_{n,1}^\trans
                \mU_{n,1} \mD_n \mV_{n,2}^\trans \\
        &\quad+
            \frac{1}{\sqrt{n}} \, \mE_{n,21}
                \mhV_{n,1} \mhD_{n,1}(k)^+ \mhU_{n,1}^\trans
                \mE_{n,12}
    \end{split} \\
    \begin{split}
        &=
            \sqrt{n} \, 
                \mU_{n,2} \mD_n 
                \mTheta_{n,1} \mhD_{n,1}(k)^+ \mPhi_{n,1}^\trans
                \mD_n \mV_{n,2}^\trans \\
        &\quad+
            \mU_{n,2} \mD_n \mTheta_{n,1} \mhD_{n,1}(k)^+ \mtE_{n,12} \\
        &\quad+
            \mtE_{n,21} 
                \mhD_{n,1}(k)^+ \mPhi_{n,1}^\trans \mD_n \mV_{n,2}^\trans \\
        &\quad+
            \frac{1}{\sqrt{n}} \, \mtE_{n,21} \mhD_{n,1}(k)^+ \mtE_{n,12},
    \end{split}
\end{align*}
where 
$\mTheta_{n,1} = \mV_{n,1}^\trans \mhV_{n,1}$,
$\mPhi_{n,1}   = \mU_{n,1}^\trans \mhU_{n,1}$,
$\mtE_{n,12} = \mhU_{n,1}^\trans \mE_{n,12}$, and
$\mtE_{n,21} = \mE_{n,21} \mhV_{n,1}$.  Note that $\mtE_{n,12}$ and $\mtE_{n,21}$ have \iid $\Normal( 0 ,\, \sigma^2 )$ entries, also independent of the other terms that make up $\mhX_{n,22}(k)$ and $\mX_{n,22}$.  By conditioning on $\mtE_{n,12}$ and $\mtE_{n,21}$, we can see that $\mhX_{22}(k)$ is in general a biased estimate of $\sqrt{n} \, \mU_{n,2} \mD_n \mV_{n,2}^\trans$.

To analyze $\mhX_{22}(k)$, we need to impose some additional assumptions.
The first assumption is fairly banal and involves the leave-out sizes.

\begin{assumption}
    There exist fixed $K, L \in (0, \infty)$ (not necessarily integers),
    such that
    $\frac{n}{n_2} = K + \oh\left(\frac{1}{\sqrt{n}}\right)$ and 
	$\frac{p}{p_2} = L + \oh\left(\frac{1}{\sqrt{n}}\right)$.
\end{assumption}

\noindent
The next assumption is not quite so innocent and involves the distribution of the factors.

\begin{assumption}\label{A:held-in-orthog}
    The departure from orthogonality for the held-in factors $\mU_{n,1}$
    and $\mV_{n,1}$ is of order $1/\sqrt{n}$.  Specifically,
    \begin{gather*}
        \sup_n 
        \E \, \Big\| 
            \sqrt{n_1} 
            \Big( 
                \mU_{n,1}^\trans \mU_{n,1} 
                -
                \frac{n_1}{n} \mI_{k_0}
            \Big)
        \Big\|_\Frob^2
        <
        \infty, \quad\text{and} \\
        \sup_n
        \E \, \Big\| 
            \sqrt{p_1} 
            \Big( 
                \mV_{n,1}^\trans \mV_{n,1} 
                -
                \frac{p_1}{p} \mI_{k_0}
            \Big)
        \Big\|_\Frob^2
        <
        \infty.
    \end{gather*}
\end{assumption}

\noindent
This assumption is there so that we can apply the theory in Chapter~\ref{C:svd-behavior} to get at the behavior of the SVD of $\mX_{n,11}$.  It is satisfied, for example, if we are performing rotated cross-validation (subsection~\ref{SS:rotated-cv}) or if the factors are generated by certain stationary processes.


\section{The SVD of the held-in block}

The first step in analyzing the BCV estimate of prediction error is to see how the SVD of $\mX_{n,11}$ behaves.  With Assumptions~\ref{A:optimal-leave-out-shapes}--\ref{A:held-in-orthog}, we can start this analysis.  Our strategy is to use Assumption~\ref{A:held-in-orthog} to apply a matrix perturbation argument in combination with Theorems~\ref{T:spiked-eigenvalue-limits} and~\ref{T:spiked-eigenvector-limits}.

We show that we can apply Theorem~\ref{T:spiked-eigenvector-limits} to 
$\sqrt{n} \mU_{n,1} \mD_n \mV_{n,1} + \mE_{n,11}$ even though $\mU_{n,1}$ and $\mV_{n,1}$ do not have orthogonal columns.  First we set
\[
	\gamma_1
		= 
		\frac{K-1}{K}
		\cdot
		\frac{L}{L-1}
		\cdot
		\gamma
\]
and note that 
\(
	\frac{n_1}{p_1} 
		= 
		\frac{n_1}{n} \cdot \frac{p}{p_1} \cdot \frac{n}{p}
		=
		\gamma_1 + \oh\left( \frac{1}{\sqrt{n_1}} \right).
\)
Next, for $1 \leq i \leq k_0$, we define
\begin{gather*}
	\mu_{1,i}
		= 
			\frac{L-1}{L} \cdot \mu_{i}, \\
	\bmu_{1,i}
		=
		\begin{cases}
			\frac{K-1}{K}
			\cdot
            \left( \mu_{1,i} + \sigma^2 \right)
            \left( 1 + \frac{\sigma^2}{\gamma \mu_{1,i}} \right)
                &\text{when $\mu_{1,i} 
							 > 
							 \frac{\sigma^2}{\sqrt{\gamma_1}}$}, \\
			\frac{K-1}{K}
			\cdot
            \sigma^2 \left( 1 + \frac{1}{\sqrt{\gamma_1}} \right)^2
                &\text{otherwise.}
        \end{cases}		
			\\
	\theta_{1,i}
		=
	        \begin{cases}
				\frac{L-1}{L}
				\cdot
				\sqrt{
		            \left( 
						1 
						- 
						\frac{\sigma^4}{ \gamma_1 \mu_{1,i}^2}
					\right)
		            \left(
						1 
						+ 
						\frac{\sigma^2}{ \gamma_1 \mu_{1,i} } 
					\right)^{-1}
				}
	            &\text{when $\mu_{1,i} 
							 > 
							 \frac{\sigma^2}{\sqrt{\gamma_1}}$,} \\
	            0
	            &\text{otherwise,}
	        \end{cases} \\
    \varphi_{1,i}
        =
            \begin{cases}
				\frac{K-1}{K}
				\cdot
				\sqrt{
	                \left( 
						1 
						- 
						\frac{\sigma^4}{ \gamma_1 \mu_{1,i}^2} 
					\right)
	                \left(
						1 
						+ 
						\frac{\sigma^2}{ \mu_{1,i} } 
					\right)^{-1}
				}
                &\text{when $\mu_{1,i} 
							 > 
							\frac{\sigma^2}{\sqrt{\gamma_1}}$,} \\
                0
                &\text{otherwise.}
            \end{cases}
\end{gather*}

\begin{proposition}
\end{proposition}



The main two ingredients are two standard results about perturbed singular value decompositions.  The first theorem is about singular values:

\begin{theorem}[Mirsky]
    If $\mA$ and $\mA + \mE$ are in $\reals^{n \times p}$, then
    \[
        \sum_{i=1}^{n \wedge p}
            \big(
                \sigma_i(\mA + \mE)
                -
                \sigma_i(\mA)
            \big)^2
        \leq
        \| \mE \|_\Frob^2,
    \]
    where $\sigma_i(\cdot)$ denotes the $i$th singular value.
\end{theorem}

\noindent
The next theorem is about singular vectors.

\begin{theorem}[Wedin]
    Let
    \[
        \left(
        \begin{matrix}
            \mU_1 & \mU_2 & \mU_3
        \end{matrix}
        \right)^\trans
        \mA
        \left(
        \begin{matrix}
            \mV_1 & \mV_2
        \end{matrix}
        \right)
        =
        \left(
        \begin{matrix}
            \mSigma_1 & 0 \\
            0         & \mSigma_2 \\
            0         & 0
        \end{matrix}
        \right)
    \]
    be a singular value decomposition of $\mA$, in which the singular
    values are not necessarily in descending order.  Let $\mtA = \mA + \mE$
    be a perturbation of $\mA$ and let
    \[
        \left(
        \begin{matrix}
            \mtU_1 & \mtU_2 & \mtU_3
        \end{matrix}
        \right)^\trans
        \mtA
        \left(
        \begin{matrix}
            \mtV_1 & \mtV_2
        \end{matrix}
        \right)
        =
        \left(
        \begin{matrix}
            \mtSigma_1 & 0 \\
            0          & \mtSigma_2 \\
            0          & 0
        \end{matrix}
        \right).
    \]
    Let $\mH$ be the matrix of canonical angles between
    $\mathcal{R}(\mU_1)$ and $\mathcal{R}(\mtU_1)$, and let
    $\mXi$ be the matrix of canonical angles between
    $\mathcal{R}(\mV_1)$ and $\mathcal{R}(\mtV_1)$.  Define
    \[
        \mR = \mA \mtV_1 - \mtU_1 \mtSigma_1
    \]
    and
    \[
        \mS = \mA^\trans \mtU_1 - \mtV_1 \mtSigma_1.
    \]
    If there is a $\delta > 0$ such that
    \[
        \min_{i,j} 
            | \sigma_i( \mtSigma_1 ) - \sigma_j( \mSigma_2 )| 
        \geq 
        \delta
    \]
    and
    \[
        \min_i \sigma_i( \mtSigma_1 ) \geq \delta,
    \]
    then
    \[
        \| \sin \mH \|_\Frob^2 + \| \sin \mXi \|_\Frob^2
        \leq
        \frac{\| \mR \|_\Frob^2 + \| \mS \|_\Frob^2}{\delta^2}.
    \]
\end{theorem}

\noindent
In the above theorem, the matrix of canonical angles between 
$\mathcal{R}(\mU_1)$ and $\mathcal{R}(\mtU_1)$, $\mH$, is a matrix whose
diagonal entries are the singular values of $\mU_1^\trans \mtU_1$ and whose
off-diagonal elements are zero.  The matrix $\sin \mH$ is interpreted as element-wise application of the $\sin$ function.  The other matrices, $\mXi$ and $\sin \mXi$, are similarly defined.